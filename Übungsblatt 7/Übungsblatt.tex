\documentclass[11pt]{article}

%Packages
\usepackage{amsfonts}	      %Mathematische Zeichen und Fonts
\usepackage{mathtools}        %Extra Mathematische Symbole
\usepackage{extarrows}	      %Extra Pfeile
\usepackage{listings}         %Codeansicht
\usepackage{scrlayer-scrpage} %Seitenkopf
\usepackage{tikz}             %tikz
\usepackage{enumitem}		  %Enumerate
\usepackage{listings}		  %Code snippets
\usepackage{amsmath}
\usepackage[normalem]{ulem}
\usepackage{tikz-qtree}

\usetikzlibrary{arrows, automata, positioning}
\pagestyle{scrheadings}

\begin{document}

%Header
\ihead{\textbf{Formale Systeme, Automaten, Prozesse \\ Übungsblatt 7} \\Tutorium 11}
\ohead{Tim Luther, 410886 \\ Til Mohr, 405959\\ Simon Michau, 406133}

%Seiteninhalt
\paragraph{Aufgabe H19}
Siehe Anhang.

\paragraph{Aufgabe H20}
\begin{enumerate}
\item Grammatik in Chomsky-Normalform
\begin{itemize}
\item  Gegeben ist folgende kontextfreie Grammatik:
\\\hspace*{6mm}$S \rightarrow (S)\mid 0 \mid 1\mid \neg S \mid S \wedge S \mid S \vee S$
\item Ersetze alle Terminalsymbole durch neue Nichtterminalsymbole:
\\\hspace*{6mm} $S \rightarrow R_{(}SR_{)}\mid R_{0} \mid R_{1}\mid R_{\neg} S \mid S R_{\wedge} S \mid S R_{\vee} S$
\\\hspace*{6mm} $R_{0} \rightarrow 0$
\\\hspace*{6mm} $R_{1} \rightarrow 1$
\\\hspace*{6mm} $R_{\wedge} \rightarrow \wedge$
\\\hspace*{6mm} $R_{\vee} \rightarrow \vee$
\\\hspace*{6mm} $R_{\neg} \rightarrow \neg$
\\\hspace*{6mm} $R_{(} \rightarrow ($
\\\hspace*{6mm} $R_{)} \rightarrow )$
\item Ersetzen von Ableitungen auf mehr als zwei Nichtterminalsymbole:
\\\hspace*{6mm} $S \rightarrow R_{(}A\mid R_{0} \mid R_{1}\mid R_{\neg} S \mid S B \mid S C$
\\\hspace*{6mm} $A \rightarrow SR_{)}$
\\\hspace*{6mm} $B \rightarrow R_{\wedge} S$
\\\hspace*{6mm} $C \rightarrow R_{\vee} S$
\\\hspace*{6mm} $R_{0} \rightarrow 0$
\\\hspace*{6mm} $R_{1} \rightarrow 1$
\\\hspace*{6mm} $R_{\wedge} \rightarrow \wedge$
\\\hspace*{6mm} $R_{\vee} \rightarrow \vee$
\\\hspace*{6mm} $R_{\neg} \rightarrow \neg$
\\\hspace*{6mm} $R_{(} \rightarrow ($
\\\hspace*{6mm} $R_{)} \rightarrow )$
\\\hspace*{6mm}
\item Einfügen neuer Produktionen:
\\\hspace*{6mm} $S \rightarrow R_{(}A\mid R_{0} \mid R_{1}\mid R_{\neg} S \mid R_{\neg} R_{0} \mid R_{\neg} R_{1} \mid S B \mid R_{0}B \mid R_{1}B \mid \hspace*{15mm} SC \mid R_{0} C \mid R_{1}C$
\\\hspace*{6mm} $A \rightarrow SR_{)} \mid R_{0}R_{)}\mid R_{1}R_{)}$
\\\hspace*{6mm} $B \rightarrow R_{\wedge} S \mid R_{\wedge}R_{0} \mid R_{\wedge}R_{1}$
\\\hspace*{6mm} $C \rightarrow R_{\vee} S \mid R_{\vee}R_{0} \mid R_{\vee}R_{1}$
\\\hspace*{6mm} $R_{0} \rightarrow 0$
\\\hspace*{6mm} $R_{1} \rightarrow 1$
\\\hspace*{6mm} $R_{\wedge} \rightarrow \wedge$
\\\hspace*{6mm} $R_{\vee} \rightarrow \vee$
\\\hspace*{6mm} $R_{\neg} \rightarrow \neg$
\\\hspace*{6mm} $R_{(} \rightarrow ($
\\\hspace*{6mm} $R_{)} \rightarrow )$
\\\hspace*{6mm}
\item Streichen der Kettenregeln und fertige Chomsky-Normalform:
\\\hspace*{6mm} $S \rightarrow R_{(}A\mid R_{\neg} S \mid R_{\neg} R_{0} \mid R_{\neg} R_{1} \mid S B \mid R_{0}B \mid R_{1}B \mid \hspace*{17mm} SC \mid R_{0} C \mid R_{1}C \mid 0\mid 1$
\\\hspace*{6mm} $A \rightarrow SR_{)} \mid R_{0}R_{)}\mid R_{1}R_{)}$
\\\hspace*{6mm} $B \rightarrow R_{\wedge} S \mid R_{\wedge}R_{0} \mid R_{\wedge}R_{1}$
\\\hspace*{6mm} $C \rightarrow R_{\vee} S \mid R_{\vee}R_{0} \mid R_{\vee}R_{1}$
\\\hspace*{6mm} $R_{0} \rightarrow 0$
\\\hspace*{6mm} $R_{1} \rightarrow 1$
\\\hspace*{6mm} $R_{\wedge} \rightarrow \wedge$
\\\hspace*{6mm} $R_{\vee} \rightarrow \vee$
\\\hspace*{6mm} $R_{\neg} \rightarrow \neg$
\\\hspace*{6mm} $R_{(} \rightarrow ($
\\\hspace*{6mm} $R_{)} \rightarrow )$
\\\hspace*{6mm}
\end{itemize}
\item Grammatik in Greibach-Normalform: Ausgangsgrammatik ist die Grammatik in CNF aus 1)
\begin{itemize}
\item Entferne Linksrekursion in S, da S über die Linksrekursiven Ableitungen SB und SC verfügt:
\\\hspace*{6mm} $S \rightarrow R_{(}A\mid R_{\neg} S \mid R_{\neg} R_{0} \mid R_{\neg} R_{1} \mid R_{0}B \mid R_{1}B \mid R_{0} C \mid R_{1}C \mid \hspace*{16mm}R_{(}AZ \mid R_{\neg} SZ \mid R_{\neg} R_{0}Z \mid R_{\neg} R_{1}Z \mid R_{0}BZ \mid R_{1}BZ \mid \hspace*{16mm}R_{0} CZ \mid R_{1}CZ$
\\\hspace*{6mm} $A \rightarrow SR_{)} \mid R_{0}R_{)}\mid R_{1}R_{)}$
\\\hspace*{6mm} $B \rightarrow R_{\wedge} S \mid R_{\wedge}R_{0} \mid R_{\wedge}R_{1}$
\\\hspace*{6mm} $C \rightarrow R_{\vee} S \mid R_{\vee}R_{0} \mid R_{\vee}R_{1}$
\\\hspace*{6mm} $R_{0} \rightarrow 0$
\\\hspace*{6mm} $R_{1} \rightarrow 1$
\\\hspace*{6mm} $R_{\wedge} \rightarrow \wedge$
\\\hspace*{6mm} $R_{\vee} \rightarrow \vee$
\\\hspace*{6mm} $R_{\neg} \rightarrow \neg$
\\\hspace*{6mm} $R_{(} \rightarrow ($
\\\hspace*{6mm} $R_{)} \rightarrow )$
\\\hspace*{6mm} $Z \rightarrow BZ \mid CZ \mid B \mid C$
\item Löse die Ableitung $A \rightarrow SR_{)}$ und die von Z auf:
\\\hspace*{6mm} $S \rightarrow R_{(}A\mid R_{\neg} S \mid R_{\neg} R_{0} \mid R_{\neg} R_{1} \mid R_{0}B \mid R_{1}B \mid R_{0} C \mid R_{1}C \mid \hspace*{16mm}R_{(}AZ \mid R_{\neg} SZ \mid R_{\neg} R_{0}Z \mid R_{\neg} R_{1}Z \mid R_{0}BZ \mid R_{1}BZ \mid \hspace*{16mm}R_{0} CZ \mid R_{1}CZ$
\\\hspace*{6mm} $A \rightarrow R_{0}R_{)}\mid R_{1}R_{)}\mid R_{(}AR_{)}\mid R_{\neg} SR_{)} \mid R_{\neg} R_{0}R_{)} \mid R_{\neg} R_{1}R_{)} \mid \hspace*{16mm}R_{0}BR_{)} \mid R_{1}BR_{)} \mid R_{0} CR_{)} \mid R_{1}CR_{)} \mid R_{(}AZR_{)} \mid R_{\neg} SZR_{)} \mid \hspace*{16mm}R_{\neg} R_{0}ZR_{)} \mid R_{\neg} R_{1}ZR_{)} \mid R_{0}BZR_{)} \mid R_{1}BZR_{)} \mid R_{0} CZR_{)} \mid \hspace*{16mm}R_{1}CZR_{)}$
\\\hspace*{6mm} $B \rightarrow R_{\wedge} S \mid R_{\wedge}R_{0} \mid R_{\wedge}R_{1}$
\\\hspace*{6mm} $C \rightarrow R_{\vee} S \mid R_{\vee}R_{0} \mid R_{\vee}R_{1}$
\\\hspace*{6mm} $R_{0} \rightarrow 0$
\\\hspace*{6mm} $R_{1} \rightarrow 1$
\\\hspace*{6mm} $R_{\wedge} \rightarrow \wedge$
\\\hspace*{6mm} $R_{\vee} \rightarrow \vee$
\\\hspace*{6mm} $R_{\neg} \rightarrow \neg$
\\\hspace*{6mm} $R_{(} \rightarrow ($
\\\hspace*{6mm} $R_{)} \rightarrow )$
\\\hspace*{6mm} $Z \rightarrow R_{\wedge} S \mid R_{\wedge}R_{0} \mid R_{\wedge}R_{1} \mid R_{\vee} S \mid R_{\vee}R_{0} \mid R_{\vee}R_{1} \mid R_{\wedge} SZ \mid \hspace*{16mm}R_{\wedge}R_{0}Z \mid R_{\wedge}R_{1}Z \mid R_{\vee} SZ \mid R_{\vee}R_{0}Z \mid R_{\vee}R_{1}Z$
\item Leite jedes erste Nichtterminal auf sein entsprechendes Terminalsymbol ab $\rightarrow$ fertige Greibach-Normalform:
\\\hspace*{6mm} $S \rightarrow (A\mid \neg S \mid \neg R_{0} \mid \neg R_{1} \mid 0B \mid 1B \mid 0 C \mid 1C \mid (AZ \mid \neg SZ \mid \hspace*{16,5mm} \neg R_{0}Z \mid \neg R_{1}Z \mid 0BZ \mid 1BZ \mid  0 CZ \mid 1 CZ\mid 0\mid 1$
\\\hspace*{6mm} $A \rightarrow 0R_{)}\mid 1R_{)}\mid (AR_{)}\mid \neg SR_{)} \mid \neg R_{0}R_{)} \mid \neg R_{1}R_{)} \mid 0BR_{)} \mid \hspace*{16,5mm} 1BR_{)} \mid 0 CR_{)} \mid 1CR_{)} \mid (AZR_{)} \mid \neg SZR_{)} \mid \neg R_{0}ZR_{)} \mid \hspace*{16,5mm}\neg R_{1}ZR_{)} \mid 0BZR_{)} \mid 1BZR_{)} \mid 0 CZR_{)} \mid 1CZR_{)}$
\\\hspace*{6mm} $B \rightarrow \wedge S \mid \wedge R_{0} \mid \wedge R_{1}$
\\\hspace*{6mm} $C \rightarrow \vee S \mid \vee R_{0} \mid \vee R_{1}$
\\\hspace*{6mm} $R_{0} \rightarrow 0$
\\\hspace*{6mm} $R_{1} \rightarrow 1$
\\\hspace*{6mm} $R_{\wedge} \rightarrow \wedge$
\\\hspace*{6mm} $R_{\vee} \rightarrow \vee$
\\\hspace*{6mm} $R_{\neg} \rightarrow \neg$
\\\hspace*{6mm} $R_{(} \rightarrow ($
\\\hspace*{6mm} $R_{)} \rightarrow )$
\\\hspace*{6mm} $Z \rightarrow \wedge S \mid \wedge R_{0} \mid \wedge R_{1} \mid \vee S \mid \vee R_{0} \mid \vee R_{1} \mid \wedge SZ \mid \wedge R_{0}Z \mid \hspace*{16,5mm} \wedge R_{1}Z \mid \vee SZ \mid \vee R_{0}Z \mid \vee R_{1}Z$

\end{itemize} 
\end{enumerate}


\paragraph{Aufgabe H21}
\begin{enumerate}[label=\alph*)]
\item Bewertung: 0/10
\begin{itemize}
\item Beweist du oder Widerlegst du die Aussage?
\item Ist "$§L_1$ nicht regulär" eine Annahme oder die Schlussfolgerung?
\item Das Pumping-Lemma gilt für ein $n\in\mathbb{N}$. Wenn du also zeigen willst, dass $L_1$ nicht regulär ist, musst du es für alle $n\in\mathbb{N}$ widerlegen.
\item Ist $\vert w \vert \geq n$?
\item 1. Wenn du schon "eine Zerlegung" schreibst, sage auch, welche Zerlegung dies ist.\\2. Siehe oben: Wenn du zeigen willst, dass $L_1$ nicht regulär ist, musst du das Pumping-Lemma für alle Zerlegungen $xyz$ mit den Bedingungen widerlegen.
\item "ungefähr" - schwacher Ausdruck. Zeige was du meinst (Berechnungen)
\item Ist $i=200$ hier?
\item Wieso ist $xy^{200}z\not\in\L_1$? Zeige warum
\item Manchmal schreibst du $L$ anstatt $L_1$!
\end{itemize}

\item Bewertung: 0/10
\begin{itemize}
\item Beweist du oder Widerlegst du die Aussage?
\item "Sei n eine große und schöne Zahl". Bitte was. Ist $n\in\mathbb{C}$ oder $n\in\mathbb{N}$? Soll $n>100$ sein? Drücke dich mathematisch aus.
\item Wieso gilt $w\in\L_2$ und $\vert w \vert \geq n$? Zeige warum es gilt.
\item Wenn du zeigen willst, dass $L_2$ nicht regulär ist, musst du das Pumping-Lemma für alle Zerlegungen $xyz$ mit den Bedingungen widerlegen.
\item Was sind Bedinungen (1) und (2)? Schreibe sie doch hin.
\item "[...] für jedes natürliche i.". Schwammig, $i\in\mathbb{N}_0$ genau.
\item Wieso gilt $((abc)^n)^i=(abc)^{in}$? Schreibe es hin. Genauso wieso $xy^iz$ diesem gleichen soll.
\item Wieso ist das gepumpte Wort in der Sprache enthalten? Für $i=0$ ist es nämlich nicht.
\item Manchmal schreibst du $L$ anstatt $L_2$!
\item Antwort ist falsch. $L_2$ ist nicht regulär.
\end{itemize}

\item Bewertung: 0/10
\begin{itemize}
\item Beweist du oder Widerlegst du die Aussage?
\item Was ist dein $n$? Ist das $n\in\mathbb{C}$?
\item Wenn du zeigen willst, dass $L_3$ nicht kontextfrei ist, musst du das Pumping-Lemma für alle Zerlegungen $uvwxy$ mit den Bedingungen widerlegen.
\item Definiere dein $i$!
\item Wieso soll das "trivialistischterweise" nicht gelten? Für $i=1$ passt doch alles! Begründe (wähle zB geeignet ein $i$ aus)
\end{itemize}
\end{enumerate}

\end{document}