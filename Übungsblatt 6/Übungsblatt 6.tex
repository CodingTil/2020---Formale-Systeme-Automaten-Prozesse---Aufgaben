\documentclass[11pt]{article}

%Packages
\usepackage{amsfonts}	      %Mathematische Zeichen und Fonts
\usepackage{mathtools}        %Extra Mathematische Symbole
\usepackage{extarrows}	      %Extra Pfeile
\usepackage{listings}         %Codeansicht
\usepackage{scrlayer-scrpage} %Seitenkopf
\usepackage{tikz}             %tikz
\usepackage{enumitem}		  %Enumerate
\usepackage{listings}		  %Code snippets
\usepackage{amsmath}
\usepackage[normalem]{ulem}
\usepackage{tikz-qtree}

\usetikzlibrary{arrows, automata, positioning}
\pagestyle{scrheadings}

\begin{document}

%%%%%%%%%% Header %%%%%%%%%%%%%%%%%%%%%%%%%%%%%%%%%%%%%%%%%%%%%%%%%%%%%%%%%%%%%
\ihead{\textbf{Formale Systeme, Automaten, Prozesse \\ Übungsblatt 6} \\Tutorium 11}
\ohead{Tim Luther, 410886 \\ Til Mohr, 405959\\ Simon Michau, 406133}

%%%%%%%%%% Seiteninhalt %%%%%%%%%%%%%%%%%%%%%%%%%%%%%%%%%%%%%%%%%%%%%%%%%%%%%%%

%%%%%%%%%%%%%%%%%%%%%%%%%%%%%%%%%%%%%%%%%%%%%%%%%%%%%%%%%%%%%%%%%%%%%%%%%%%%%%%
%
%			Aufgabe H16
%
%%%%%%%%%%%%%%%%%%%%%%%%%%%%%%%%%%%%%%%%%%%%%%%%%%%%%%%%%%%%%%%%%%%%%%%%%%%%%%%
\paragraph{Aufgabe H16}
$(ab+bc+ca)^* \cap (abc+bab+cba)^* \cap (abca+bc+caba)^*$ ist der Schnitt aus drei Sprachen. Bilde NFAs für diese 3 Sprachen:
\begin{center}
\begin{tikzpicture}[->, >=latex, node distance = 2cm, semithick]
\node[initial,state,accepting]	(0)										{0};
\node[state]			(1)	[right=2cm of 0]					{1};
\node[state]			(2)	[above=2cm of 0]					{2};
\node[state]			(3) [below=2cm of 0]					{3};

\path 	(0)	edge [below,bend right=20]	node {a} (1)
		(1) edge [above,bend right=20]	node {b} (0)
		(0) edge [right,bend right=20]	node {b} (2)
		(2) edge [left,bend right=20]	node {c} (0)
		(0) edge [right,bend left=20]	node {c} (3)
		(3) edge [left,bend left=20]	node {a} (0)
;
\end{tikzpicture}

\begin{tikzpicture}[->, >=latex, node distance = 2cm, semithick]
\node[initial,state,accepting]	(0)											{0};
\node[state]			(1)	[below right=0.75cm and 2cm  of 0]		{1};
\node[state]			(2)	[above right=0.75cm and 2cm of 0]		{2};
\node[state]			(3)	[above right=2cm and 0.75cm of 0]		{3};
\node[state]			(4)	[above left=2cm and 0.75cm of 0]		{4};
\node[state]			(5) [below right=2cm and 0.75cm of 0]		{5};
\node[state]			(6) [below left=2cm and 0.75cm of 0]		{6};

\path 	(0)	edge [below,bend left=20]	node {a} (1)
		(1) edge [right,bend right=20]	node {b} (2)
		(2) edge [above,bend left=20]	node {c} (0)
		(0) edge [right,bend left=20]	node {b} (3)
		(3) edge [above,bend right=20]	node {a} (4)
		(4) edge [left,bend left=20]	node {b} (0)
		(0) edge [right,bend right=20]	node {c} (5)
		(5) edge [below,bend left=20]	node {b} (6)
		(6) edge [left,bend right=20]	node {a} (0)
;
\end{tikzpicture}

\begin{tikzpicture}[->, >=latex, node distance = 2cm, semithick]
\node[initial,state,accepting]	(0)										{0};
\node[state]			(1)	[below right=0.75cm and 2cm of 0]	{1};
\node[state]			(2)	[right=3.5cm of 0]					{2};
\node[state]			(3)	[above right=0.75cm and 2cm of 0]	{3};
\node[state]			(4)	[above=2cm of 0]					{4};
\node[state]			(5)	[below right=2cm and 0.75cm of 0]	{5};
\node[state]			(6) [below=3.5cm of 0]					{6};
\node[state]			(7) [below left=2cm and 0.75cm of 0]	{7};

\path 	(0)	edge [above,bend left=20]	node {a} (1)
		(1) edge [above,bend left=20]	node {b} (2)
		(2) edge [below,bend left=20]	node {c} (3)
		(3) edge [below,bend left=20]	node {a} (0)
		(0) edge [right,bend right=20]	node {b} (4)
		(4) edge [left,bend right=20]	node {c} (0)
		(0) edge [left,bend right=20]	node {c} (5)
		(5) edge [left,bend right=20]	node {a} (6)
		(6) edge [right,bend right=20]	node {b} (7)
		(7) edge [right,bend right=20]	node {a} (0)
;
\end{tikzpicture}
\end{center}
Aus diesen 3 NFAs kann man nun einen Produktautomaten bilden (den Schnitt):
\begin{center}
\begin{tikzpicture}[->, >=latex, node distance = 2cm, semithick]
\node[initial,state,accepting]	(0)												{0,0,0};
\node[state]			(111)	[below right=1.75cm and 1cm of 0]		{1,1,1};
\node[state]			(022)	[right=3cm of 111]						{0,2,2};
\node[state]			(303)	[above right=1.75cm and 1cm of 022]		{3,0,3};
\node[state]			(010)	[above left=1.75cm and 1cm of 303]		{0,1,0};
\node[state]			(224)	[left=3cm of 010]						{2,2,4};
\node[state]			(234)	[above=2cm of 0]						{2,3,4};
\node[state]			(355)	[below=2cm of 0]						{3,5,5};

\path	(0) 	edge [above]	node {a} (111)
		(111)	edge [above]	node {b} (022)
		(022)	edge [above]	node {c} (303)
		(303)	edge [above]	node {a} (010)
		(010)	edge [above]	node {b} (224)
		(224)	edge [above]	node {c} (0)
		(0)		edge [left]		node {b} (234)
		(0)		edge [left]		node {c} (355)
;
\end{tikzpicture}
\end{center}
Minimiert ergibt das:
\begin{center}
\begin{tikzpicture}[->, >=latex, node distance = 2cm, semithick]
\node[initial,state,accepting]	(0)												{0,0,0};
\node[state]			(111)	[below right=1.75cm and 1cm of 0]		{1,1,1};
\node[state]			(022)	[right=3cm of 111]						{0,2,2};
\node[state]			(303)	[above right=1.75cm and 1cm of 022]		{3,0,3};
\node[state]			(010)	[above left=1.75cm and 1cm of 303]		{0,1,0};
\node[state]			(224)	[left=3cm of 010]						{2,2,4};

\path	(0) 	edge [above]	node {a} (111)
		(111)	edge [above]	node {b} (022)
		(022)	edge [above]	node {c} (303)
		(303)	edge [above]	node {a} (010)
		(010)	edge [above]	node {b} (224)
		(224)	edge [above]	node {c} (0)
;
\end{tikzpicture}
\end{center}
Daraus folgt, dass $(ab+bc+ca)^* \cap (abc+bab+cba)^* \cap (abca+bc+caba)^* = (abcabc)^*$ ist.

%%%%%%%%%%%%%%%%%%%%%%%%%%%%%%%%%%%%%%%%%%%%%%%%%%%%%%%%%%%%%%%%%%%%%%%%%%%%%%%
%
%			Aufgabe H17
%
%%%%%%%%%%%%%%%%%%%%%%%%%%%%%%%%%%%%%%%%%%%%%%%%%%%%%%%%%%%%%%%%%%%%%%%%%%%%%%%
\paragraph{Aufgabe H17}
\begin{enumerate}[label=\alph*)]
\item \begin{itemize}
\item $S$ ist das Startsymbol, und ist daher auch erreichbar.
\item Da es die Ableitungs $S\Rightarrow BDA$ gibt, ist $A$ erreichbar.
\item Da es die Ableitungs $S\Rightarrow BDA$ gibt, ist $B$ erreichbar.
\item Da es die Ableitungs $S\Rightarrow DC$ gibt, ist $C$ erreichbar.
\item Da es die Ableitungs $S\Rightarrow DC$ gibt, ist $D$ erreichbar.
\end{itemize}
Damit sind alle $X\in N$ erreichbar. Die Menge der unerreichbaren Symbole von G ist also $\O$.
\item \begin{itemize}
\item Das Nichtterminal $A$ ist produktiv, da es z.B. die Ableitung $A\Rightarrow a$ gibt.
\item Das Nichtterminal $B$ ist produktiv, da es z.B. die Ableitung $B\Rightarrow \epsilon$ gibt.
\item Das Nichtterminal $C$ ist produktiv, da es z.B. die Ableitung $C\Rightarrow a$ gibt.
\item Das Nichtterminal $D$ ist produktiv, da es z.B. die Ableitung $D\Rightarrow aA$ gibt, und $A$ selber auch produktiv ist.
\item Das Nichtterminal $S$ ist produktiv, da es z.B. die Ableitung $S\Rightarrow DC$ gibt, und sowohl $C$ als auch $D$ selber produktiv sind.
\end{itemize}
Damit ist die Menge der unproduktiven Symbole von G gleich $\O$.
\item Nein, $L(G)=\O$ gilt nicht, da es z.B. die Ableitung $S\Rightarrow BDA\Rightarrow \epsilon DA\Rightarrow aAA\Rightarrow aaA \Rightarrow aaa$ gibt. Damit enthält $L(G)$ also mind. ein Wort und ist somit ungleich $\O$.
\item Damit $\epsilon\in L(G)$ gilt, muss es eine Ableitung $S\xRightarrow[]{*}\epsilon$ geben. Da nur $B$ eine Ableitung nach $\epsilon$ enthält (siehe Teilaufgabe e)), muss auch $S\xRightarrow[]{*} B\Rightarrow\epsilon$ gelten. $S$ hingegen kann nicht direkt auf $B$ ableiten, und da es keine unproduktiven Symbole in G gibt, gilt $S\xRightarrow[]{*}\epsilon$ nicht, wodurch $\epsilon\not\in L(G)$ gilt.
\item Erstelle $pre^{*}_{G}(\epsilon)$:
\begin{center}
\begin{tikzpicture}[->, >=latex, node distance = 2cm, semithick]
\node[initial,state,accepting] (0) {0};
\end{tikzpicture}
\end{center}
\begin{center}
\begin{tikzpicture}[->, >=latex, node distance = 2cm, semithick]
\node[initial,state,accepting] (0) {0};
\path (0) edge[loop,above] node{B} (0);
\end{tikzpicture}
\end{center}
Also ist die Menge der nullierbaren Symbole $\{B\}$.
\item Da es keine unnerreichbaren oder unproduktiven Symbole in G gibt, muss man G nur bezüglich den nullierbaren Symbolen - also $B$ - anpassen. Die einfachste Möglichkeit ist, alle Vorkommen von B in den Ableitungen anzupassen:\\
$G': S\rightarrow DA \mid ScDA \mid cDDA \mid ADDA \mid DAC \mid DC$
\\\hspace*{6mm} $A\rightarrow SA \mid aA \mid a \mid d \mid b$
\\\hspace*{6mm} $C\rightarrow aD \mid dA \mid bC \mid a \mid d \mid b$
\\\hspace*{6mm} $D\rightarrow S \mid ScS \mid cDS \mid ADS \mid cC \mid aA$
\end{enumerate}


%%%%%%%%%%%%%%%%%%%%%%%%%%%%%%%%%%%%%%%%%%%%%%%%%%%%%%%%%%%%%%%%%%%%%%%%%%%%%%%
%
%			Aufgabe H18
%
%%%%%%%%%%%%%%%%%%%%%%%%%%%%%%%%%%%%%%%%%%%%%%%%%%%%%%%%%%%%%%%%%%%%%%%%%%%%%%%
\paragraph{Aufgabe H18}
\begin{enumerate}
\item $G_1: S\rightarrow (S)\mid S\wedge S\mid S\vee S\mid \neg S \mid 0 \mid 1$
\item $G_2: S\rightarrow R\wedge R\mid R\vee F\mid F\vee R\mid R\vee R\mid \neg F\mid A$
\\\hspace*{6mm} $R \rightarrow (R\wedge R)\mid (R\vee F)\mid (F\vee R)\mid (R\wedge R)\mid \neg (F)\mid A$
\\\hspace*{6mm} $F \rightarrow (F\wedge F)\mid (F\wedge R)\mid (R\wedge F)\mid (F\vee F)\mid B\mid \neg (R)$
\\\hspace*{6mm} $A \rightarrow \neg 0\mid 1$
\\\hspace*{6mm} $B \rightarrow \neg 1\mid 0$
\item Ableitungsbaum für $\neg 0 \wedge 1$:
\\ \ \\
\begin{tikzpicture}
\tikzset{edge from parent/.style={draw,edge from parent path={(\tikzparentnode.south)-- +(0,-8pt)-| (\tikzchildnode)}}}
\tikzset{frontier/.style={distance from root=90pt}}
\Tree 	[.S
			[.R 
				[.A 
					[.$\neg 0$ ]
				]	
			]			
			[.$\wedge$ $\wedge$ ]
			[.R
				[.A 
					[.1 ]
				]
			]		
		]
]
\end{tikzpicture}
\\ \ \\
Ableitungsbaum für $\neg(\neg 1 \wedge 0)$:
\\ \ \\
\begin{tikzpicture}
\tikzset{edge from parent/.style={draw,edge from parent path={(\tikzparentnode.south)-- +(0,-8pt)-| (\tikzchildnode)}}}
\tikzset{frontier/.style={distance from root=150pt}}
\Tree 	[.S 
			[.$\neg$ $\neg$ ]
			[.F 
				[.( ( ]
				[.F 
					[.B 
						[$\neg 1$ ]
					]
				]		
				[.$\wedge$ $\wedge$ ]
				[.F 
					[.B
						[.0 ]
					]
				]		
				[.) ) ]
			]
		]
]
\end{tikzpicture}
\end{enumerate}
Die Grammatik aus 2. ist im Allgemeinen in Terme aufgeteilt, die zu 1 auswerten (R) und solche, die zu 0 auswerten (F). Da jeder Ausdruck der kontextfreien Grammatik zu 1 auswerten soll, sind im Folgenden die Ergebnisse von allen UND beziehungsweise ODER Verknüpfungen aus wahren Termen (R) und unwahren Termen (F), sowie der Negation, tabellarisch dargestellt:
\\ \ \\
\begin{tabular}[h]{c|cc}
$\wedge$ & R & F \\
\hline
R & R & F\\
F & F & F\\
\end{tabular}
\quad
\begin{tabular}[h]{c|cc}
$\vee$ & R & F \\
\hline
R & R & R\\
F & R & F\\
\end{tabular}
\quad
\begin{tabular}[h]{c|c}
$\neg$ \\
\hline
R & F\\
F & R\\
\end{tabular}
\\ \ \\
Die Grammatik besteht aus 5 Nichtterminalsymbolen, von denen sich einzig A und B direkt auf Terminalsymbole ableiten lassen. A erzeugt hierbei Terminalsymbole, die 1 ergeben und B erzeugt diejenigen, die 0 ergeben. Des Weiteren gibt es die Nichterminalsymbole R und F. R erzuegt alle Terme aus kleineren Teiltermen, die zu 1 evaluiert werden mithilfe der logischen Verknüpfungen(UND/ODER/NEG) aus der Tabelle. Analog dazu erzeugt das Nichtterminalsymbol F alle logischen Terme, die zu 0 auswerten. Das Startsymbol S lässt sich entweder direkt zu A ableiten oder es erzeugt alle Verknüpfungen der logischen Tabelle, die den Wahrheitswert 1 haben. Somit ist gewährleistet, dass die Grammatik einzig Ausdrücke erzeugt, die zu 1 evalueirt werden.   
\\Auf diese Art lassen sich alle Grammatikregeln nach Wahrheitswert trennen. Startet man also mit einem S als Startsymbol lassen sich nach den Regeln der Aussagenlogik nur Ausdrücke ableiten deren Wahrheitswert ebenfalls 1 entspricht.
\\Betrachten wir nun einige Beispiele um dies zu zeigen:

\begin{itemize}
\item[Bsp. 1:] $0$ sollte nicht durch die Grammatik erzeugbar sein:
\\$S\rightarrow$ Scheitert, da die einzige Möglichkeit von S ein einzelnes Terminalsymbol abzuleiten, die Ableitung nach A ist. A wird ausschließlich zu Terminalsymbolen evaluiert, deren Wahrheitswert in der Booleschen Algebra 1 entspricht. Bei allen anderen Ableitungen für S hat man mindestens zwei Terminalsymbole, die somit nicht zur 0 evaluiert werden können.  

\item[Bsp. 2:] $0\wedge 1$ sollte nicht durch die Grammatik erzeugbar sein:
\\$S\rightarrow R \wedge R \rightarrow $ Scheitert, weil R nicht nach 0 abgeleitet werden kann, aus dem selben Grund wie bei 1).

\item[Bsp. 3:] $1\wedge (0\vee 0)$ sollte nicht durch die Grammatik erzeugbar sein:
\\Hierbei ist die äußerste Verknüpfung $\wedge$. Da die Grammatik die Terme von außen nach innen konstruiert, ist die einzig mögliche Ableitung $S \rightarrow R\wedge R$.  
\\Versuche somit : $S\rightarrow R\wedge R \rightarrow A \wedge R\rightarrow 1 \wedge R\rightarrow 1 \wedge (R\vee F) \rightarrow 1 \wedge (R\vee B) \rightarrow 1 \wedge (R\vee 0)$
\\Da R unter keinen Umständen nach 0 abgeleitet werden kann, scheitert dieser Ableitungsversuch. 

\item[Bsp. 4:] $1 \vee (0\wedge \neg 0)$ sollte erzeugbar sein:
\\$S \rightarrow R \vee F \rightarrow A \vee F \rightarrow A \vee (F\wedge R) \rightarrow 1 \vee (F \wedge R) \rightarrow 1 \vee (0 \wedge R) \rightarrow 1 \vee (0 \wedge A) \rightarrow 1 \vee (0\wedge \neg 0)$ \hfill $\Box$

\item[Bsp. 5:] $1 \wedge (0\vee 1)$, sollte erzeugbar sein:
\\$S \rightarrow R \wedge R \rightarrow R \wedge (F \vee R) \rightarrow A \wedge (F \vee R) \rightarrow 1 \wedge (F \vee R)\rightarrow 1 \wedge (B \vee R) \rightarrow 1 \wedge (0 \vee R) \rightarrow 1 \wedge (0 \vee A) \rightarrow 1 \wedge (0 \vee 1)$
\item[Bsp. 6:] $\neg (1 \wedge (0\vee \neg1))$, sollte erzeugbar sein:
\\$S \rightarrow \neg F \rightarrow \neg (R \wedge F) \rightarrow \neg (A \wedge F) \rightarrow \neg (1 \wedge F) \rightarrow \neg (1 \wedge (F \vee F)) \rightarrow \neg (1 \wedge (B \vee F)) \rightarrow \neg (1 \wedge (0 \vee F)) \rightarrow \neg (1 \wedge (0 \vee B)) \rightarrow \neg (1 \wedge (0 \vee \neg 1))$

\end{itemize} 


\end{document}