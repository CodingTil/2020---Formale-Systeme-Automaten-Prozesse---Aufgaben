\documentclass[11pt]{article}

%Packages
\usepackage{amsfonts}	      %Mathematische Zeichen und Fonts
\usepackage{mathtools}        %Extra Mathematische Symbole
\usepackage{extarrows}	      %Extra Pfeile
\usepackage{listings}         %Codeansicht
\usepackage{scrlayer-scrpage} %Seitenkopf
\usepackage{tikz}             %tikz
\usepackage{enumitem}		  %Enumerate
\usepackage{listings}		  %Code snippets
\usepackage{amsmath}
\usepackage[normalem]{ulem}
\usepackage{tikz-qtree}

\usetikzlibrary{arrows, automata, positioning}
\pagestyle{scrheadings}

\begin{document}

%%%%%%%%%% Header %%%%%%%%%%%%%%%%%%%%%%%%%%%%%%%%%%%%%%%%%%%%%%%%%%%%%%%%%%%%%
\ihead{\textbf{Formale Systeme, Automaten, Prozesse \\ Übungsblatt 6} \\Tutorium 11}
\ohead{Tim Luther, 410886 \\ Til Mohr, 405959\\ Simon Michau, 406133}

%%%%%%%%%% Seiteninhalt %%%%%%%%%%%%%%%%%%%%%%%%%%%%%%%%%%%%%%%%%%%%%%%%%%%%%%%

%%%%%%%%%%%%%%%%%%%%%%%%%%%%%%%%%%%%%%%%%%%%%%%%%%%%%%%%%%%%%%%%%%%%%%%%%%%%%%%
%
%			Aufgabe H16
%
%%%%%%%%%%%%%%%%%%%%%%%%%%%%%%%%%%%%%%%%%%%%%%%%%%%%%%%%%%%%%%%%%%%%%%%%%%%%%%%
\paragraph{Aufgabe H16}


%%%%%%%%%%%%%%%%%%%%%%%%%%%%%%%%%%%%%%%%%%%%%%%%%%%%%%%%%%%%%%%%%%%%%%%%%%%%%%%
%
%			Aufgabe H17
%
%%%%%%%%%%%%%%%%%%%%%%%%%%%%%%%%%%%%%%%%%%%%%%%%%%%%%%%%%%%%%%%%%%%%%%%%%%%%%%%
\paragraph{Aufgabe H17}
\begin{enumerate}[label=\alph*)]
\item
\item
\item
\item
\item
\item
\end{enumerate}


%%%%%%%%%%%%%%%%%%%%%%%%%%%%%%%%%%%%%%%%%%%%%%%%%%%%%%%%%%%%%%%%%%%%%%%%%%%%%%%
%
%			Aufgabe H18
%
%%%%%%%%%%%%%%%%%%%%%%%%%%%%%%%%%%%%%%%%%%%%%%%%%%%%%%%%%%%%%%%%%%%%%%%%%%%%%%%
\paragraph{Aufgabe H18}
\begin{enumerate}
\item $G_1: S\rightarrow (S)\mid S\wedge S\mid S\vee S\mid \neg S \mid 0 \mid 1$
\item $G_2: S\rightarrow (S)\mid S\wedge S\mid S\vee F\mid F\vee S\mid S\vee S\mid \neg F\mid 1$
\\\hspace*{6mm} $F \rightarrow (F) \mid F \wedge S\mid S \wedge F\mid F \wedge F\mid F \vee F \mid \neg S\mid 0$ 
\item Ableitungsbaum für $\neg 0 \wedge 1$:
\\ \ \\
\begin{tikzpicture}
\tikzset{edge from parent/.style={draw,edge from parent path={(\tikzparentnode.south)-- +(0,-8pt)-| (\tikzchildnode)}}}
\tikzset{frontier/.style={distance from root=90pt}}
\Tree 	[.S 
			[.S
				[.$\neg$ $\neg$ ]
				[.F 0 ]	
			]
			[.$\wedge$ $\wedge$ ]
			[.S 1 ]
		]
]
\end{tikzpicture}
\\ \ \\
Ableitungsbaum für $\neg(\neg 1 \wedge 0)$:
\\ \ \\
\begin{tikzpicture}
\tikzset{edge from parent/.style={draw,edge from parent path={(\tikzparentnode.south)-- +(0,-8pt)-| (\tikzchildnode)}}}
\tikzset{frontier/.style={distance from root=150pt}}
\Tree 	[.S 
			[.$\neg$ $\neg$ ]
			[.F 
				[.( ( ]
				[.F 
					[.F 
						[.$\neg$ $\neg$ ]					
						[.S 1 ]					
					]
					[.$\wedge$ $\wedge$ ]
					[.F 0 ]					
				]
				[.) ) ]				
			]	
		]
]
\end{tikzpicture}
\end{enumerate}
Die Grammatik aus 2. kommt mit 2 Nichtterminalsymbolen aus, von denen eines das Startsymbol ist. Hierbei gilt, dass es ein Nichtterminal für Ausdrücke gibt die zu 0 bzw. false auswerten, und ein Nichtterminal für Ausdrücke die zu 1 auswerten (wahr). Dazu betrachten wir für die logischen Operatoren die folgenden Wahrheitstabellen mit S=1 und F=0:
\\ \ \\
\begin{tabular}[h]{c|cc}
$\wedge$ & S & F \\
\hline
S & S & F\\
F & F & F\\
\end{tabular}
\quad
\begin{tabular}[h]{c|cc}
$\vee$ & S & F \\
\hline
S & S & S\\
F & S & F\\
\end{tabular}
\quad
\begin{tabular}[h]{c|c}
$\neg$ \\
\hline
S & F\\
F & S\\
\end{tabular}
\\ \ \\Aus den Tabellen können wir nun folgern, dass es für $\wedge$ diese Zuweisungen geben muss:
\begin{itemize}
\item $S \rightarrow S \wedge S$
\item $F \rightarrow S \wedge F \mid F \wedge S \mid F \wedge F$  
\end{itemize}
Für $\vee$ muss es diese Zuweisungen geben:
\begin{itemize}
\item $S \rightarrow S \vee S \mid S \vee F \mid F \vee S$
\item $F \rightarrow F \vee F$
\end{itemize} 
Und für $\neg$:
\begin{itemize}
\item $S \rightarrow \neg F$
\item $F \rightarrow \neg S$
\end{itemize}
Zusätzlich gibt es noch die trivialen Zuweisungen $F\rightarrow (F)$ und $S\rightarrow (S)$, sowie die Terminalsymbole 0 und 1.
\\Auf diese Art lassen sich alle Grammatikregeln nach Wahrheitswert trennen. Startet man also mit einem S als Startsymbol lassen sich nach den Regeln der Aussagenlogik nur Ausdrücke ableiten deren Wahrheitswert ebenfalls 1 entspricht.
\\Betrachten wir nun einige Beispiele um dies zu zeigen:

\begin{itemize}
\item[Bsp. 1:] $0$ sollte nicht durch die Grammatik erzeugbar sein:
\\$S\rightarrow$ Scheitert, da keine Zuweisung auf ein Nichtterminalsymbol getätigt werden kann, die nicht gleichzeitig die Anzahl der Zeichen erhöht.

\item[Bsp. 2:] $0\wedge 1$ sollte nicht durch die Grammatik erzeugbar sein:
\\$S\rightarrow S \wedge S \rightarrow S \wedge 1 \rightarrow$ Scheitert, aus selbem Grund wie Bsp. 1

\item[Bsp. 3:] $0\wedge (1\vee 0)$ sollte nicht durch die Grammatik erzeugbar sein:
\\Versuche: $S\rightarrow S\wedge S \rightarrow S \wedge (S)\rightarrow S \wedge (S\vee F)\rightarrow S \wedge (1\vee 0)$
\\Scheitert, da 0 nicht durch S erzeugbar ist (siehe Bsp. 1).
\\Versuche $S \rightarrow S \vee S \rightarrow S \wedge S \vee S$
\\Scheitert, da korrekte Klammerung nicht erzeugbar ist und 0 wieder nicht durch S erzeugbar ist
\\Da dies die einzigen beiden Wege sind aus S einen Ausdruck mit genau einem $\wedge$ und genau einem $\vee$ zu erzeugen, ist es unmöglich diesen Ausdruck zu erzeugen.

\item[Bsp. 4:] $1 \vee (0\wedge \neg 0)$ sollte erzeugbar sein:
\\$S \rightarrow S \vee F \rightarrow 1 \vee F \rightarrow 1 \vee (F) \rightarrow 1 \vee (F \wedge S) \rightarrow 1 \vee (0 \wedge S) \rightarrow 1 \vee (0 \wedge \neg F) \rightarrow 1 \vee (0\wedge \neg 0)$ \hfill $\Box$

\item[Bsp. 5:] $1 \wedge 0 \vee 1$, sollte erzeugbar sein:
\\$S \rightarrow S \wedge S \rightarrow S \wedge F \vee S \rightarrow 1 \wedge 0 \vee 1$

\end{itemize} 


\end{document}