\documentclass[11pt]{article}

%Packages
\usepackage{amsfonts}	      %Mathematische Zeichen und Fonts
\usepackage{mathtools}        %Extra Mathematische Symbole
\usepackage{extarrows}	      %Extra Pfeile
\usepackage{listings}         %Codeansicht
\usepackage{scrlayer-scrpage} %Seitenkopf
\usepackage{tikz}             %tikz
\usepackage{enumitem}		  %Enumerate
\usepackage{listings}		  %Code snippets
\usepackage{amsmath}
\usepackage[normalem]{ulem}

\usetikzlibrary{arrows, automata, positioning}
\pagestyle{scrheadings}

\begin{document}

%%%%%%%%%% Header %%%%%%%%%%%%%%%%%%%%%%%%%%%%%%%%%%%%%%%%%%%%%%%%%%%%%%%%%%%%%
\ihead{\textbf{Formale Systeme, Automaten, Prozesse \\ Übungsblatt 4} \\Tutorium 11}
\ohead{Tim Luther, 410886 \\ Til Mohr, 405959\\ Simon Michau, 406133}

%%%%%%%%%% Seiteninhalt %%%%%%%%%%%%%%%%%%%%%%%%%%%%%%%%%%%%%%%%%%%%%%%%%%%%%%%

%%%%%%%%%%%%%%%%%%%%%%%%%%%%%%%%%%%%%%%%%%%%%%%%%%%%%%%%%%%%%%%%%%%%%%%%%%%%%%%
%
%			Aufgabe H10
%
%%%%%%%%%%%%%%%%%%%%%%%%%%%%%%%%%%%%%%%%%%%%%%%%%%%%%%%%%%%%%%%%%%%%%%%%%%%%%%%
\paragraph{Aufgabe H10}
\begin{center}
%Verschönert
\begin{tikzpicture}[->, >=latex, node distance = 2cm, semithick]
\node[initial,state]	(1)											{$\{1\}$};
\node[state]			(15) 	[above right=1cm and 0.5cm of 1]	{$\{1,5\}$};
\node[state,accepting]	(8) 	[below right=1cm and 0.5cm of 1]	{$\{8\}$};
\node[state,accepting]	(48) 	[below right=1cm and 0.5cm of 15]	{$\{4,8\}$};
\node[state]			(3) 	[below right=1cm and 0.5cm of 48]	{$\{3\}$};
\node[state,accepting]	(7) 	[above right=1cm and 0.5cm of 3]	{$\{7\}$};
\node[state,accepting]	(26) 	[right=0.75cm of 7]					{$\{2,6\}$};
\node[state,accepting]	(46) 	[right=0.75cm of 26]				{$\{4,6\}$};
\node[state]			(4) 	[right=0.75cm of 46]				{$\{4\}$};
\node[state]			(0) 	[above right=1cm and 0.5cm of 48]	{$\O$};

\path 	(1) 	edge [above left]			node {a} (15)
				edge [below left] 			node {b} (8)
		(15) 	edge [loop above=20, above] node {a} (15)
				edge [above right] 			node {b} (48)
		(8) 	edge [bend right=45, right] node {a} (0)
				edge [bend right=30, below] node {b} (3)
		(48) 	edge [left] 				node {a} (0)
				edge [below] 				node {b} (3)
		(3) 	edge [right]			 	node {a} (7)
				edge [bend right=20, below]	node {b} (4)
		(7) 	edge [right] 				node {a} (0)
				edge [below] 				node {b} (26)
		(26) 	edge [below]		 		node {a} (46)
				edge [bend right=10, above]	node {b} (0)
		(46) 	edge [below] 				node {a} (4)
				edge [bend right=20, above]	node {b} (0)
		(4) 	edge [bend right=30, above] node {a,b} (0)
		(0) 	edge [loop above=20, above]	node {a,b} (0)
;
\end{tikzpicture}
\end{center}


Erstelle Zustandspaartabelle und streiche alle Paare, bei denen der eine Zustand Endzustand ist und der andere nicht, da diese Paare eindeutig unterscheidbar sind. Die Zellen unter der der Diagonalen ignorieren wir, da sie symmetrisch sind.\\ \ \\
\begin{tabular}[h]{l|c|c|c|c|c|c|c|c|c|c}
 	&\{1\} &\{1,5\} &\{2,6\} &\{3\} &\{4\} &\{4,6\} &\{4,8\} &\{7\} &\{8\} &$\emptyset$\\
\hline
 \{1\} & $\cdot$ & &X& & &X&X&X&X& \\
\hline
\{1,5\}& & $\cdot$ &X& & &X&X&X&X& \\
\hline
\{2,6\}& & & $\cdot$ &X&X& & & & &X\\
\hline
 \{3\} & & & & $\cdot$ & &X&X&X&X& \\
\hline
 \{4\} & & & & & $\cdot$ &X&X&X&X& \\
\hline
\{4,6\}& & & & & & $\cdot$ & & & &X\\
\hline
\{4,8\}& & & & & & & $\cdot$ & & &X\\
\hline
 \{7\} & & & & & & & & $\cdot$ & &X\\
\hline
 \{8\} & & & & & & & & & $\cdot$ &X\\
\hline
$\emptyset$& & & & & & & & & & $\cdot$\\
\end{tabular}
\\ \ \\

Erstelle nun Übergangstabelle mit allen Paaren die noch nicht als unterscheidbar bekannt sind und streiche alle Zustandspaare die für a oder b auf einem unterscheidbaren Zustand landen.
\\ \ \\
\begin{tabular}{ll|ll|ll}
$Z_{1}$&$Z_{2}$&a&&b&\\
\hline
(\{1\}, & \{1,5\}) & (\{1,5\}, & \{1,5\}) & (\{8\}, & \{4,8\})\\
{(\{1\},} & \{3\}) & (\{1,5\}, & \{7\}) & \\
(\{1\}, & \{4\})   & (\{1,5\}, & \{$\emptyset$\}) & (\{8\}, & \{$\emptyset$\})\\
(\{1\}, & \{$\emptyset$\}) & (\{1,5\}, & \{$\emptyset$\}) & (\{8\}, & \{$\emptyset$\})\\
(\{1,5\}, & \{3\}) & (\{1,5\}, & \{7\}) & \\
(\{1,5\}, & \{4\}) & (\{1,5\}, & \{$\emptyset$\}) & (\{4,8\}, & \{$\emptyset$\})\\
(\{1,5\}, & \{$\emptyset$\}) & (\{1,5\}, & \{$\emptyset$\}) & (\{4,8\}, & \{$\emptyset$\})\\
(\{2,6\}, & \{4,6\}) & (\{4,6\}, & \{4\}) & \\
(\{2,6\}, & \{4,8\}) & (\{4,6\}, & \{$\emptyset$\}) & \\
(\{2,6\}, & \{7\}) & (\{4,6\}, & \{$\emptyset$\}) & \\
(\{2,6\}, & \{8\}) & (\{4,6\}, & \{$\emptyset$\}) & \\
(\{3\}, & \{4\}) & (\{7\}, & \{$\emptyset$\}) & \\
(\{3\}, & \{$\emptyset$\}) & (\{7\}, & \{$\emptyset$\}) & \\
(\{4\}, & \{$\emptyset$\}) & (\{$\emptyset$\}, & \{$\emptyset$\}) & (\{$\emptyset$\}, & \{$\emptyset$\})\\
(\{4,6\}, & \{4,8\}) & (\{4\}, & \{$\emptyset$\}) & (\{$\emptyset$\}, & \{3\})\\
(\{4,6\}, & \{7\}) & (\{4\}, & \{$\emptyset$\}) & (\{$\emptyset$\}, & \{2,6\})\\
(\{4,6\}, & \{8\}) & (\{4\}, & \{$\emptyset$\}) & (\{$\emptyset$\}, & \{3\})\\
(\{4,8\}, & \{7\}) & (\{$\emptyset$\}, & \{$\emptyset$\}) & (\{3\}, & \{2,6\})\\
(\{4,8\}, & \{8\}) & (\{$\emptyset$\}, & \{$\emptyset$\}) & (\{3\}, & \{3\})\\
(\{7\}, & \{8\}) & (\{$\emptyset$\}, & \{$\emptyset$\}) & (\{2,6\}, & \{3\})\\
\end{tabular}
\\ \ \\
Somit sind die ununterscheidbaren Zustandspaare:
\\ \ \\
\begin{tabular}{ll|ll|ll}
$Z_{1}$&$Z_{2}$&a&&b&\\
\hline
(\{1\}, & \{1,5\}) & (\{1,5\}, & \{1,5\}) & (\{8\}, & \{4,8\})\\
(\{4\}, & \{$\emptyset$\}) & (\{$\emptyset$\}, & \{$\emptyset$\}) & (\{$\emptyset$\}, & \{$\emptyset$\})\\
(\{4,8\}, & \{8\}) & (\{$\emptyset$\}, & \{$\emptyset$\}) & (\{3\}, & \{3\})\\
\end{tabular}
\\ \ \\
und es ergibt sich die finale Zustandspaartabelle, in der o für ein ununterscheidbares Zustandspaar steht: 
\\ \ \\
\begin{tabular}[h]{l|c|c|c|c|c|c|c|c|c|c}
 	&\{1\} &\{1,5\} &\{2,6\} &\{3\} &\{4\} &\{4,6\} &\{4,8\} &\{7\} &\{8\} &$\emptyset$\\
\hline
 \{1\} & $\cdot$ &o&X&X&X&X&X&X&X&X\\
\hline
\{1,5\}& & $\cdot$ &X&X&X&X&X&X&X&X\\
\hline
\{2,6\}& & & $\cdot$ &X&X&X&X&X&X&X\\
\hline
 \{3\} & & & & $\cdot$ &X&X&X&X&X&X\\
\hline
 \{4\} & & & & & $\cdot$ &X&X&X&X&o\\
\hline
\{4,6\}& & & & & & $\cdot$ &X&X&X&X\\
\hline
\{4,8\}& & & & & & & $\cdot$ &X&o&X\\
\hline
 \{7\} & & & & & & & & $\cdot$ &X&X\\
\hline
 \{8\} & & & & & & & & & $\cdot$ &X\\
\hline
$\emptyset$& & & & & & & & & & $\cdot$\\
\end{tabular}
\\ \ \\
Erstelle nun minimalen deterministischen Automaten (mit \{\{1\},\{1,5\}\}=\{1,5\}, \{\{8\},\{4,8\}\}=\{4,8\} und \{\{4\},\{$\emptyset$\}\}=\{4\}):
\\ \ \\
\begin{tikzpicture}[->, >=latex, node distance = 2cm, semithick]
\node[initial,state]	(15)						{$\{1,5\}$};
\node[state,accepting]	(48) 	[right=1cm of 15]	{$\{4,8\}$};
\node[state]			(3) 	[right=1cm of 48]	{$\{3\}$};
\node[state,accepting]	(7) 	[right=1cm of 3]	{$\{7\}$};
\node[state,accepting]	(26) 	[right=1cm of 7]	{$\{2,6\}$};
\node[state,accepting]	(46) 	[right=1cm of 26]	{$\{4,6\}$};
\node[state]			(4) 	[below=1cm of 48]	{$\{4\}$};
s
\path 	(15) 	edge [loop above=20, above] node {a} (15)
				edge [above]	 			node {b} (48)
		(48) 	edge [left] 				node {a} (4)
				edge [above] 				node {b} (3)
		(3) 	edge [above]			 	node {a} (7)
				edge [bend left=10, above]	node {b} (4)
		(7) 	edge [bend left=15, above]	node {a} (4)
				edge [above] 				node {b} (26)
		(26) 	edge [above]		 		node {a} (46)
				edge [bend left=20, above]	node {b} (4)
		(46) 	edge [bend left=25, above]	node {a,b} (4)
		(4) 	edge [loop left=20, left] 	node {a,b} (4)
;
\end{tikzpicture}

%%%%%%%%%%%%%%%%%%%%%%%%%%%%%%%%%%%%%%%%%%%%%%%%%%%%%%%%%%%%%%%%%%%%%%%%%%%%%%%
%
%			Aufgabe H11
%
%%%%%%%%%%%%%%%%%%%%%%%%%%%%%%%%%%%%%%%%%%%%%%%%%%%%%%%%%%%%%%%%%%%%%%%%%%%%%%%
\paragraph{Aufgabe H11}
\begin{enumerate}[label=\alph*)]
\item $L_1=\{a^{n}|\sqrt{n}\in \mathbb{N}, n>100\}=\{a^{121}, a^{144}, a^{169}, ...\}=\{a^{m^2}|m\in\mathbb{N},m>10\}$\\
Sei $m\in\mathbb{N},m>10$ gegeben. Dann ist $w\coloneqq a^{m^2} \in L_1$. Dann kann man $w$ zu $xyz$ zerlegen, mit $|xy|\leq m, y \neq \epsilon$.
\[m^2=|xyz|<|xy^2z|=|xyz|+|y|\leq m^2+m\]
Jedoch ist das nach $w$ in $L_1$ nächstgrößere Wort nur \[(m+1)^2=m^2+2m+1\] groß.\\
Damit ist klar, dass $xy^2z \not\in L_1$ ist. Dadurch gilt das Pumping-Lemma nicht, weshalb $L_1$ keine reguläre Sprache ist.

\item 
\end{enumerate}

%%%%%%%%%%%%%%%%%%%%%%%%%%%%%%%%%%%%%%%%%%%%%%%%%%%%%%%%%%%%%%%%%%%%%%%%%%%%%%%
%
%			Aufgabe H12
%
%%%%%%%%%%%%%%%%%%%%%%%%%%%%%%%%%%%%%%%%%%%%%%%%%%%%%%%%%%%%%%%%%%%%%%%%%%%%%%%
\paragraph{Aufgabe H12}
\begin{center}
 \hline
 \multicolumn{4}{|c|}{Regex Matcher} \\
 \hline
 Language & GoLang & Java & Java\\
 \hline
 Algorithm & NFA & NFA & backtracking\\
 \hline
 1	&	$79 ms$	&	$139 ms$	&	$8 ms$\\
 2	&	$132 ms$	&	$116 ms$	&	$4 ms$\\
 3	&	$153 ms$	&	$145 ms$	&	$8 ms$\\
 4	&	$21 ms$	&	$124 ms$	&	$15 ms$\\
 5	&	$23 ms$	&	$162 ms$	&	$30 ms$\\
 6	&	$101 ms$	&	$128 ms$	&	$60 ms$\\
 7	&	$81 ms$	&	$136 ms$	&	$90 ms$\\
 8	&	$29 ms$	&	$173 ms$	&	$49 ms$\\
 9	&	$31 ms$	&	$164 ms$	&	$96 ms$\\
 10	&	$54 ms$	&	$169 ms$	&	$171 ms$\\
 11	&	$35 ms$	&	$146 ms$	&	$197 ms$\\
 12	&	$35 ms$	&	$151 ms$	&	$373 ms$\\
 13	&	$56 ms$	&	$362 ms$	&	$668 ms$\\
 14	&	$40 ms$	&	$179 ms$	&	$1253 ms$\\
 15	&	$42 ms$	&	$147 ms$	&	$2109 ms$\\
 16	&	$43 ms$	&	$193 ms$	&	$3648 ms$\\
 17	&	$88 ms$	&	$121 ms$	&	$6160 ms$\\
 18	&	$95 ms$	&	$120 ms$	&	$10304 ms$\\
 19	&	$76 ms$	&	$110 ms$	&	$16812 ms$\\
 20	&	$163 ms$	&	$72 ms$	&	$27152 ms$\\
 \hline
 Average Time	&	$68 ms$	&	$152 ms$	&	$3460 ms$\\
 \hline
\end{tabular}
\end{center}
\begin{enumerate}[label=\alph*)]
\item Man kann erkennen, dass GoLang, welce NFAs verwendet, in etwa für alle $i \in \{1...20\}$ in etwa gleich schnell arbeitet. Die Ausreißer lassen sich beisielsweise durch Unterbrechung des Programms durch andere Programm erklären.\\
Java hingegen benutzt backtracking. Er versucht also jedes Zeichen des Inputs mit dem Regex zu matchen. Falls dies nicht geht, wird Java also die letzten durchläufe zurückgehen und einen anderen Weg einschlagen. Für kleine Eingaben ($i \in \{1...5\}$) ist dies sehr schnell, aber mit wachsendem Input wird es exponentiell aufwendiger.

\item Die Unterschiede in der Laufzeit kommen daher, dass beide Sprachen verschiedene Ansätze haben, um herauszufinden, ob ein String einem Regex matcht. Wie in $a)$ beschrieben, benutzt golang NFA und ist daher für alle Eingaben hier relativ schnell. Java benutzt standartmäßig backtracking, was für wachsenden Input exponentiell länger zu brauchen scheint. Backtracking versucht quasi, die Eingabe in einen Baum aufzuteilen (nach dem gegebenen Regex). Wenn ein Zweig fehlschlät, wird in einem anderen Zweig weitergearbeitet.

\item Die erste Java-Spalte in der Tabelle ist die Umsetzung des Regex mithilfe der NFA-Implementierung von letzter Woche. Damit benutzt sie wie GoLang auch einen NFA. Man stellt fest, dass hier für alle $i \in \{1...20\}$ das Programm auch etwa gleich schnell arbeitet, doch ca. $100 ms$ langsamer als die GoLang Implementierung. Dies könnte verschiedene Gründe haben, beispielsweise dass GoLang direkt zu Byte-Code compiled wird, Java nicht. Eventuell kann unsere Java NFA-Implementierung auch noch optimiert werden, wodurch auch Laufzeitunterschiede zu erklären sind.
\end{enumerate}

\end{document}