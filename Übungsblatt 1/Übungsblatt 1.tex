\documentclass[11pt]{article}

%Packages
\usepackage{amsfonts}	      %Mathematische Zeichen und Fonts
\usepackage{mathtools}        %Extra Mathematische Symbole
\usepackage{extarrows}	      %Extra Pfeile
\usepackage{listings}         %Codeansicht
\usepackage{scrlayer-scrpage} %Seitenkopf
\usepackage{tikz}             %tikz

\usetikzlibrary{arrows, automata, positioning}
\pagestyle{scrheadings}

\begin{document}

%Header
\ihead{\textbf{Formale Systeme, Automaten, Prozesse \\ Übungsblatt 1} \\Tutorium 11}
\ohead{Tim Luther, 410886 \\ Til Mohr, 405959\\ Simon Michau, 406133}

%Seiteninhalt
\paragraph{Aufgabe H1}
Gegeben ist $v,w\in\Sigma^{*}, vw=w^{R}v, \left\lvert w \right\lvert \geq  \left\lvert v \right\lvert$. 
\\Daraus kann man schließen, es gibt ein $c\in\Sigma^{*}$ mit $w^{R}=vc$ und $w=cv$. Denn dann gilt auch $w=xv=(w^{R})^{R}=(vx)^{R}=x^{R}v^{R}$. Da $\left\lvert c \right\lvert = \left\lvert c^{R} \right\lvert$ und $\left\lvert v \right\lvert = \left\lvert v^{R} \right\lvert$ gilt, folgt $c=c^{R}$ und $v=v^{R}$.
\\Daraus kann man folgern, dass
\begin{center}
$(vw)^{R}=w^{R}v^{R}=w^{R}v=vw$
\end{center}
gilt.
\hfill $\Box$

\paragraph{Aufgabe H2}
Wir konstruieren einen Automaten, der das Problem modelliert. Dabei stehen a,b und c für den Übergang zum jeweiligen Zielzustand.
%Automat
\begin{center}
\begin{tikzpicture}[->, >=latex, node distance = 2.8cm, semithick]
\node[initial,state]	(A)									{A};
\node[state]			(B)	[below left=2cm and 1cm of A]	{B};
\node[state,accepting]	(C) [below right=2cm and 1cm of A]	{C};

\path 	(A) edge [bend left=20, left] 	node {b} (B)
			edge [bend left=20, right] 	node {c} (C)
	 	(B)	edge [bend left=20, left] 	node {a} (A)
	 		edge [bend left=20, below] 	node {c} (C)
	 	(C)	edge [bend left=20, right] 	node {a} (A)
	 		edge [bend left=20, below] 	node {b} (B)
;
\end{tikzpicture}
\end{center}
Wir reduzieren nun diesen Automaten um einfacher einen regulären Ausdruck für den Automaten ablesen zu können.

%reduzierter Automat
\begin{center}
\begin{tikzpicture}[->, >=latex, node distance = 2.8cm, semithick]
\node[initial,state]	(A)				{A};
\node[state,accepting]	(C)	[right=2cm] {C};

\path 	(A)	edge [loop above] 			node {ba+ca}(A)
			edge [bend left=20, above]	node {c+bc}	(C)
		(C) edge [bend left=20, below]	node {a+ba}	(A)
			edge [loop right]			node {ac+bc}(C)
;
\end{tikzpicture}
\end{center}	
Damit der Automat wie in der Aufgabenstellung gewünscht funktioniert, nehmen wir an dass es der Anfangszustand A durch einen nicht eingezeichneten Übergang \textbf{a} erreicht wird.
\\Vom Zustand A aus kann der Automat zunächst beliebig oft zwischen einem anderen Raum und A hin- und herspringen, solange er am Ende wieder im Zustand A angelangt \textbf{(ba+ca)*},
\\allerdings muss er irgendwann durch den Übergang \textbf{(c+bc)} zum Zustand C gelangen, weil dies der einzige gültige Endzustand ist.
\\nun kann der Automat entweder im Zustand C verbleiben und somit enden, oder
\\beliebig oft durch den Übergang \textbf{(a+ba)} zurück nach A springen, solange er letztenendes wieder durch \textbf{(c+bc)} wieder im Endzustand landet. Hierbei kann der Automat wieder beliebig oft den Übergang \textbf{(ba+ca)} verwenden, wenn er sich im Zustand A befindet.
\\Setzt man nun alle Ausdrücke zusammen, erhält man den Ausdruck
\begin{center}
A (BA+CA)* (C+BC) (AC+BC+(A+BA)(BA+CA)*(C+BC))*
\end{center}
welcher jeden nicht leeren Pfad durch das Museum beschreibt.

\paragraph{Aufgabe H3}
Es gilt wieder der Homomorphismus $h:a \mapsto b, b \mapsto ab$.
\\Wenden wir also nun $h$ an um die Sprache zu bestimmen: 
\begin{center}
h(\{a,b\}*) = h(\{a,b\})* $\overset{h}{=}$ \{b,ab\}*
\end{center}
Als regulärer Ausdruck ergibt sich also (b+ab)*, die Sprache aus den Alphabetsymbolen a und b, in denen das Unterwort aa nicht vorkommt. 
\\ \\Leite nun regulären Ausdruck für h(h(\{a,b\}*)) her. 
\begin{center}
h(h(\{a,b\}*)) = h(\{b,ab\}*) = h(\{b,ab\})* $\overset{h}{=}$ \{ab,bab\}*
\end{center}
Als regulärer Ausdruck also (ab+bab)*.

\end{document}