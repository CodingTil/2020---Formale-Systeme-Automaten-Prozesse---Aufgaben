\documentclass[11pt]{article}

%Packages
\usepackage{amsfonts}	      %Mathematische Zeichen und Fonts
\usepackage{mathtools}        %Extra Mathematische Symbole
\usepackage{extarrows}	      %Extra Pfeile
\usepackage{listings}         %Codeansicht
\usepackage{scrlayer-scrpage} %Seitenkopf
\usepackage{tikz}             %tikz
\usepackage{enumitem}		  %Enumerate
\usepackage{listings}		  %Code snippets
\usepackage{amsmath}
\usepackage[normalem]{ulem}
\usepackage{tikz-qtree}

\usetikzlibrary{arrows, automata, positioning}
\pagestyle{scrheadings}

\begin{document}

%Header
\ihead{\textbf{Formale Systeme, Automaten, Prozesse \\ Übungsblatt 8} \\Tutorium 11}
\ohead{Tim Luther, 410886 \\ Til Mohr, 405959\\ Simon Michau, 406133}

%Seiteninhalt
\paragraph{Aufgabe H30}
\begin{enumerate}[label=\arabic*)]
\item Die Sprache ist kontextfrei:
\begin{center}
\begin{tikzpicture}[->, >=latex, node distance = 2cm, semithick]
\node[initial,state]	(0)											{};
\node[state]			(A1)	[above right=1cm and 2cm of 0]		{A1};
\node[state]			(A2)	[right=2cm of A1]					{A2};
\node[state]			(B1) 	[below=2cm of A1]					{B1};
\node[state]			(B2)	[right=2cm of B1]					{B2};
\node[state,accepting]	(1)		[below right=1cm and 2cm of A2]		{};

\path	(0) 	edge [above] node {a,$\Gamma_0 \mid$n$\Gamma_0$} (A1)
		(0) 	edge [below] node {b,$\Gamma_0 \mid$n$\Gamma_0$} (B1)
		(A1)	edge [above] node {a,n$\mid \epsilon$} (A2)
		(A1)	edge [left] node {b,X$\mid$nX} (B1)
		(A1)	edge [loop above, above] node {a,X$\mid$nX} (A1)
		(B1)	edge [below] node {b,n$\mid \epsilon$} (B2)
		(B1)	edge [loop below, below] node {b,X$\mid$nX} (B1)
		(A2)	edge [above] node {a,$\Gamma_0 \mid \Gamma_0$} (1)
		(A2)	edge [left] node {b,n$\mid \epsilon$} (B2)
		(A2)	edge [loop above, above] node {a,n$\mid \epsilon$} (A2)
		(B2)	edge [below] node {b,$\Gamma_0 \mid \Gamma_0$} (1)
		(B2)	edge [loop below, below] node {b,n$\mid \epsilon$} (B2)
;
\end{tikzpicture}
\end{center}

\item Angenommen $L_2$ sei kontextfrei. Dann muss das Pumpinglemma für kontextfreie Sprachen gelten:
\\Sei $n\in\mathbb{N}$, $z\in L_2$ mit $z=a^{2^n}$ und $\vert z \vert = 2^n \geq n$.
\\Dann muss es eine Zerlegung von $z$ in $uvwxy$ geben mit:
\begin{enumerate}[label=(\arabic*)]
\item $\vert vwx \vert \leq n$
\item $vert vx \vert > 0$
\item $uv^iwx^iy\in L_2 \forall i\in\mathbb{N}$
\end{enumerate}
Betrachte $i=2$:
\[\vert uv^2wx^2y \vert = \vert uvwxy \vert + \vert v \vert + \vert x \vert = \vert z \vert + \vert vx \vert \leq 2^n+n < 2^{n+1}\]
Damit ist $uv^2wx^2y\not\in L_2$ also gilt das Pumpinglemma nicht und $L_2$ ist nicht kontextfrei.
\item

\end{enumerate}

\paragraph{Aufgabe H31}
%nicht sicher ob das so funktioniert, sorry musste den Stoff schnell nachholen :/
\begin{itemize}
\item[a)]
\begin{center}
		\begin{tikzpicture}[->, >=latex, node distance = 2cm, semithick]
			\node[initial,state]	(0)										{};
			\node[state]			(q0)	[above right=1cm and 2cm of 0]	{$q_0$};
			\node[state]			(q1)	[right=2cm of q0]				{$q_1$};
			\node[state]			(q2)	[below right=1cm and 2cm of 0] 	{$q_2$};
			\node[state]			(q3)	[right=2cm of q2] 				{$q_3$};	
			\node[state,accepting]	(q4) 	[right=7cm of 0]				{$q_4$};

\path	(0) 	edge [above left] node {$\epsilon$, $X \mid X$} 			(q0)
			 	edge [below left] node {$\epsilon$, $X \mid X$} 			(q2)
		(q0)	edge [loop above] node[align=left] {$a, X \mid aaaX$\\
										$b, X \mid X$\\
										$c, X \mid X$} 						(q0)
				edge [above]	  node {$\epsilon, X \mid X$}				(q1)
		(q1)	edge [loop above] node[align=left] {$a, X \mid X$\\
										$b, a \mid \epsilon$\\
										$c, a \mid \epsilon$} 				(q1)
				edge [above right]node {$\epsilon, \Gamma_0 \mid \Gamma_0$} (q4)
		(q2)	edge [loop below] node[align=left] {$a, X \mid X$\\
										$b, X \mid X$\\
										$c, X \mid cX$}						(q2)
				edge [below]	  node {$\epsilon, X \mid X$}				(q3)
		(q3)	edge [loop below] node[align=left] {$a, cc \mid \epsilon$\\
										$b, X \mid X$\\
										$c, X \mid X$} 						(q3)
				edge [below right]node {$\epsilon, \Gamma_0 \mid \Gamma_0$}	(q4)	
;
\end{tikzpicture}
\end{center}
\end{itemize}


\paragraph{Aufgabe H32}

\end{document}