\documentclass[11pt]{article}

%Packages
\usepackage{amsfonts}	      %Mathematische Zeichen und Fonts
\usepackage{mathtools}        %Extra Mathematische Symbole
\usepackage{extarrows}	      %Extra Pfeile
\usepackage{listings}         %Codeansicht
\usepackage{scrlayer-scrpage} %Seitenkopf
\usepackage{tikz}             %tikz
\usepackage{enumitem}		  %Enumerate
\usepackage{listings}		  %Code snippets
\usepackage{amsmath}
\usepackage{stix}			  %Shuffle Symbol
\usepackage[normalem]{ulem}
\usepackage{tikz-qtree}

\usetikzlibrary{arrows, automata, positioning}
\pagestyle{scrheadings}

\begin{document}

%Header
\ihead{\textbf{Formale Systeme, Automaten, Prozesse \\ Übungsblatt 11} \\Tutorium 11}
\ohead{Tim Luther, 410886 \\ Til Mohr, 405959\\ Simon Michau, 406133}

%Seiteninhalt
\paragraph{Aufgabe H36} Die unsynchronisierten und synchronisierten Produkte:
\begin{table}[h]
\begin{tabular}{|c|c|}
 \hline
 \multicolumn{1}{|l|}{$M_1 \circ M_2$:} & \multicolumn{1}{|l|}{$M_1 \shuffle M_2$:} \\
 \begin{tikzpicture}[->, >=latex, node distance = 2cm, semithick]
	\node[state] (00) 					{0,0};
	\node[state] (10) [right=1.5cm of 00] {1,0};
	
	\path	(00)edge [loop left, left] 			node {b} (00)
				edge [bend right=20, below] 	node {a} (10)
			(10)edge [bend right=20, above]		node {b} (00); 
 \end{tikzpicture}
 &  
 \begin{tikzpicture}[->, >=latex, node distance = 2cm, semithick]
	\node[state] (00) 					{0,0};
	\node[state] (10) [right=1.5cm of 00] {1,0};
	
	\path	(00)edge [loop left, left] 			node {b} (00)
				edge [bend right=20, below] 	node {a} (10)
			(10)edge [bend right=20, above]		node {b} (00)
				edge [loop right, right]		node {b} (10); 
 \end{tikzpicture}\\
 \hline
 \multicolumn{1}{|l|}{$M_1 \circ M_3$:} & \multicolumn{1}{|l|}{$M_1 \shuffle M_3$:} \\
 \begin{tikzpicture}[->, >=latex, node distance = 2cm, semithick]
	\node[state] (00) 						{0,0};
	\node[state] (01) [right=1.5cm of 00]	{0,1};
	\node[state] (10) [below=1.5cm of 00] 	{1,0};
	\node[state] (11) [right=1.5cm of 10] 	{1,1};
 	
 	\path 	(00)edge [above] 					node {b} (01)
 				edge [left]						node {a} (10)
 			(01)edge [right]					node {a} (11)
 			(10)edge [above]					node {b} (01);
 \end{tikzpicture}
 &
 \begin{tikzpicture}[->, >=latex, node distance = 2cm, semithick]
	\node[state] (00) 						{0,0};
	\node[state] (01) [right=1.5cm of 00]	{0,1};
	\node[state] (10) [below=1.5cm of 00] 	{1,0};
	\node[state] (11) [right=1.5cm of 10] 	{1,1};
 	
 	\path 	(00)edge [loop left, left]			node {b} (00)
 				edge [above] 					node {b} (01)
 				edge [bend right, left]			node {a} (10)
 			(01)edge [loop right, right]		node {b} (01)
 				edge [bend left, right]			node {a} (11)
 			(10)edge [above]					node {b} (11)
 				edge [bend right, right]		node {b} (00)
 			(11)edge [bend left, left]			node {b} (01);
 \end{tikzpicture}\\
 \hline
 \multicolumn{1}{|l|}{$M_1 \circ M_4$:} & \multicolumn{1}{|l|}{$M_1 \shuffle M_4$:} \\
 \begin{tikzpicture}[->, >=latex, node distance = 2cm, semithick]
	\node[state] (00) 						{0,0};
	\node[state] (01) [right=1.5cm of 00]	{0,1};
	\node[state] (11) [below=1.5cm of 01] 	{1,1};
 	
 	\path 	(00)edge [loop left] 					node {b} (00)
 				edge [above]						node {c} (01)
 			(01)edge [loop right]					node {b} (01)
 				edge [bend right, left]				node {a} (11)
 			(11)edge [bend right, right]			node {b} (01);
 \end{tikzpicture}
 & 
 \begin{tikzpicture}[->, >=latex, node distance = 2cm, semithick]
	\node[state] (00) 						{0,0};
	\node[state] (01) [right=1.5cm of 00]	{0,1};
	\node[state] (10) [below=1.5cm of 00] 	{1,0};
	\node[state] (11) [right=1.5cm of 10] 	{1,1};
 	
 	\path 	(00)edge [loop left, left]			node {b} (00)
 				edge [above] 					node {c} (01)
 				edge [bend right, left]			node {a} (10)
 			(01)edge [loop right, right]		node {a,b} (01)
 				edge [bend left, right]			node {a} (11)
 			(10)edge [above]					node {c} (11)
 				edge [bend right, right]		node {b} (00)
 			(11)edge [bend left, left]			node {b} (01)
 				edge [loop right]				node {a} (11);
 \end{tikzpicture}\\
 \hline
\end{tabular}
\end{table}
\newpage

\paragraph{Aufgabe H37} Modelliere zuerst Automaten für $P_1$ und $P_2$:
\begin{table}[h]
\begin{tabular}{cc}
 \begin{tikzpicture}[->, >=latex, node distance = 2cm, semithick]
	\node[initial,state] (S) 				{$S_1$};
	\node[state] (A) [right=2cm of S]		{$A_1$};
	\node[state] (B) [below=1cm of S] 		{$B_1$};
	\node[state] (E) [below=1cm of B]		{$E_1$};
 	
 	\path 	(S)	edge [bend right, below]node {$u=0?$} 	(A)
 				edge [left]				node {$u=1?$}	(B)
 			(A)	edge [bend right, above]node {$u:=1$}	(S)
 			(B) edge [left]				node {print}	(E);
 \end{tikzpicture}
 &
 \begin{tikzpicture}[->, >=latex, node distance = 2cm, semithick]
	\node[initial,state] (S) 				{$S_2$};
	\node[state] (A) [right=2cm of S]		{$A_2$};
	\node[state] (B) [below=1cm of S] 		{$B_2$};
	\node[state] (E) [below=1cm of B]		{$E_2$};
 	
 	\path 	(S)	edge [bend right, below]node {$u=1?$} 	(A)
 				edge [left]				node {$u=0?$}	(B)
 			(A)	edge [bend right, above]node {$u:=0$}	(S)
 			(B) edge [left]				node {print}	(E);
 \end{tikzpicture}\\
\end{tabular}
\end{table}





\paragraph{Aufgabe H38} 

\end{document}