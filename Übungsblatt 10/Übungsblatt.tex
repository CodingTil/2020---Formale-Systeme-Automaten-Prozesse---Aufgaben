\documentclass[11pt]{article}

%Packages
\usepackage{amsfonts}	      %Mathematische Zeichen und Fonts
\usepackage{mathtools}        %Extra Mathematische Symbole
\usepackage{extarrows}	      %Extra Pfeile
\usepackage{listings}         %Codeansicht
\usepackage{scrlayer-scrpage} %Seitenkopf
\usepackage{tikz}             %tikz
\usepackage{enumitem}		  %Enumerate
\usepackage{listings}		  %Code snippets
\usepackage{amsmath}
\usepackage[normalem]{ulem}
\usepackage{tikz-qtree}

\usetikzlibrary{arrows, automata, positioning}
\pagestyle{scrheadings}

\begin{document}

%Header
\ihead{\textbf{Formale Systeme, Automaten, Prozesse \\ Übungsblatt 10} \\Tutorium 11}
\ohead{Tim Luther, 410886 \\ Til Mohr, 405959\\ Simon Michau, 406133}

%Seiteninhalt
\paragraph{Aufgabe H33} Sei $L$ eine kontextfreie Sprache. Dann gibt es eine Grammatik $G$ in CNF, sodass $L=L(G)$. Dann hat $G$ also nur Produktionen der Form:
\begin{align*}
A &\rightarrow BC \\
A &\rightarrow a
\end{align*}
$L^R$ ist definiert als $\{w^R \mid w \in L\}$.
\\Beschreibe $G^R$, wieder als CNF, wie folgt:
\begin{align*}
A \rightarrow BC \in G &\Rightarrow A \rightarrow CB \in G^R \\
A \rightarrow a \in G &\Rightarrow A \rightarrow a \in G^R
\end{align*}
Zeige nun: $L^R=L(G^R)$
Beweis per Induktion über Anzahl der Ableitungen:
\begin{itemize}
\item[I.A.] 
\begin{enumerate}[label=\arabic*)]
\item Sei $A \rightarrow a \in G$. Dann ist $(A \rightarrow a)^R = A \rightarrow a^R = A \rightarrow a \in G^R$.
\item Sei $A \rightarrow BC \in G$ und $B\rightarrow b, C \rightarrow c \in G \cap G^R$.
\\Dann ist $(A \rightarrow BC)^R = A \rightarrow B^R C^R \in G^R; B,C \in G \\\Leftrightarrow A \rightarrow CB \in G^R; B,C \in G^R$.
\end{enumerate}
\item[I.V.] Die Behauptung gelte für eine beliebige, aber feste Anzahl an Ableitungen.
\item[I.S.] Sei $X \rightarrow YZ \in G$ und $Y \rightarrow AB, Z \rightarrow CD \in G; A,B,C,D \in G$ mit $Y \rightarrow BA, Z \rightarrow DC \in G^R; A,B,C,D \in G^R$.
\\Dann gilt $(X \rightarrow YZ)^R = X \rightarrow Z^R Y^R \in G^R; YZ \in G \\\Leftrightarrow X \rightarrow ZY \in G^R; YZ \in G^R$
\end{itemize}
Damit gilt $L^R=L(G^R)$, wodurch $L^R$ durch eine CNF beschrieben werden kann, wodurch $L^R$ kontextfrei ist, wodurch die kontextfreien Sprachen unter \textit{Spiegelung} abgeschossen sind.


\paragraph{Aufgabe H34}


\paragraph{Aufgabe H35} Folgende Grammatik mit Startsymbol $S$ beschreibt $L$:
\begin{align*}
S &\rightarrow \epsilon \mid S'	& S' &\rightarrow ABCS' \mid ABC	    \\\\
//AB &\rightarrow BA 	& //BA &\rightarrow AB	\\
AB &\rightarrow YB 		& BA &\rightarrow YA	\\
YB &\rightarrow YZ		& YA &\rightarrow YZ	\\
YZ &\rightarrow YA		& YZ &\rightarrow YB	\\
YA &\rightarrow  BA		& YB &\rightarrow AB	\\\\
//BC &\rightarrow CB 	& //CB &\rightarrow BC	\\
BC &\rightarrow WC		& CB &\rightarrow CX	\\
WC &\rightarrow WX		& CX &\rightarrow WX	\\
WX &\rightarrow CX		& WX &\rightarrow WC	\\
XC &\rightarrow BC		& WC &\rightarrow BC	\\\\
//AC &\rightarrow CA 	& //CA &\rightarrow AC	\\
AC &\rightarrow UC		& CA &\rightarrow CV	\\
UC &\rightarrow UV		& CV &\rightarrow UV	\\
UV &\rightarrow CV		& UV &\rightarrow UC	\\
CV &\rightarrow CA		& UC &\rightarrow AC	\\\\
A &\rightarrow a								\\
B &\rightarrow b								\\
C &\rightarrow c								\\
\end{align*}


\end{document}