\documentclass[11pt]{article}

%Packages
\usepackage{amsfonts}	      %Mathematische Zeichen und Fonts
\usepackage{mathtools}        %Extra Mathematische Symbole
\usepackage{extarrows}	      %Extra Pfeile
\usepackage{listings}         %Codeansicht
\usepackage{scrlayer-scrpage} %Seitenkopf
\usepackage{tikz}             %tikz
\usepackage{enumitem}		  %Enumerate
\usepackage{listings}		  %Code snippets
\usepackage{amsmath}
\usepackage[normalem]{ulem}

\usetikzlibrary{arrows, automata, positioning}
\pagestyle{scrheadings}

\begin{document}

%%%%%%%%%% Header %%%%%%%%%%%%%%%%%%%%%%%%%%%%%%%%%%%%%%%%%%%%%%%%%%%%%%%%%%%%%
\ihead{\textbf{Formale Systeme, Automaten, Prozesse \\ Übungsblatt 5} \\Tutorium 11}
\ohead{Tim Luther, 410886 \\ Til Mohr, 405959\\ Simon Michau, 406133}

%%%%%%%%%% Seiteninhalt %%%%%%%%%%%%%%%%%%%%%%%%%%%%%%%%%%%%%%%%%%%%%%%%%%%%%%%

%%%%%%%%%%%%%%%%%%%%%%%%%%%%%%%%%%%%%%%%%%%%%%%%%%%%%%%%%%%%%%%%%%%%%%%%%%%%%%%
%
%			Aufgabe H13
%
%%%%%%%%%%%%%%%%%%%%%%%%%%%%%%%%%%%%%%%%%%%%%%%%%%%%%%%%%%%%%%%%%%%%%%%%%%%%%%%
\paragraph{Aufgabe H13}
Da wir wissen, dass $l$ eine reguläre Sprache ist, besitzt die Myhill-Nerode-Relation von $L$ nur einen endlichen Index. Dies bedeutet, dass es ab einer bestimmten Wortlänge $n\in\mathbb{L}$ keine neuen Äquivalenzklassen gibt. Desweiteren muss es von jeder Äquivalenklassen einen Repräsentatanten geben, sodass alle bezüglich der Wortlänge benachbarte Repräsentanten sich maximal in der Länge eines Buchstabens unterscheiden. Wäre dies nicht der Fall, so könnte man schließlich nicht mithilfe des Pumping-Lemmas einen Buchstaben pumpen, weswegen $L$ nicht regulär wäre. Mithilfe dieser beiden Bedingungen kann man einen Algorithmus zum Finden aller Repräsentanten aufstellen. Es genügt, Schritt für Schritt die Wortlänge zu erhöhen, und dann für jedes Wort dieser Länge in $L$ zu schauen, ob es hierfür schon einen Repräsentanten gibt (1. Bedingung). Wenn für eine Wortlänge kein Repräsentant gefunden werden konnte, gibt es auch nach der 2. Bedingung keinen weiteren Repräsentanten in längeren Wörtern:
\begin{lstlisting}[escapeinside=`']
repraesentanten=`$\O$'
for (`$n\in\mathbb{N}$') {
	gefunden=false
	for (`$w\in L $' `$\wedge$' `$ \vert w \vert = n$') {
		if (`$\neg(\exists v\in$' repraesentanten `$\wedge$' `$v\sim w)$') {
			gefunden=true
			repraesentanten=repraesentanten`$\cup \{w\}$'
		}
	}
	if (`$\neg$'gefunden) break
}
return repraesentanten
\end{lstlisting}

%%%%%%%%%%%%%%%%%%%%%%%%%%%%%%%%%%%%%%%%%%%%%%%%%%%%%%%%%%%%%%%%%%%%%%%%%%%%%%%
%
%			Aufgabe H14
%
%%%%%%%%%%%%%%%%%%%%%%%%%%%%%%%%%%%%%%%%%%%%%%%%%%%%%%%%%%%%%%%%%%%%%%%%%%%%%%%
\paragraph{Aufgabe H14}
%Schritt 1
Wir schreiben die Endzustände als Zustände mit einer $\epsilon$-Transition zum Endzustand End, und den Startzustand als Zustand zu dem eine $\epsilon$-Transition vom Startzustand Start hinführt. 
\\ \ \\
\begin{tikzpicture}[->, >=latex, node distance = 1cm, semithick]
\node[initial,state]	(START)						{Start};
\node[state]			(1) 	[right=2cm of START]{$1$};
\node[state]			(2) 	[right=2cm of 1]	{$2$};
\node[state]			(3) 	[below=2cm of 1]	{$3$};
\node[state]			(4) 	[below=2cm of 2]	{$4$};
\node[state,accepting]	(END) 	[right=2cm of 2]	{End};
s
\path 	(START) edge [above]					node {$\epsilon$} 	(1)
		(1) 	edge [above]					node {a} 			(2)
		(2)		edge [right]					node {b}			(4)
				edge [loop above, above]		node {b}			(2)
				edge [bend left=10, below right]node {c}			(3)
				edge [above]					node {$\epsilon$}	(END)
		(3)		edge [bend left=10, above left]	node {a}		(2)
				edge [left]						node {b}			(1)
				edge [bend right =65, below]	node {$\epsilon$}	(END)
		(4)		edge [below]					node {a}			(3)
;
\end{tikzpicture}
\\ \ \\
%Schritt 2
Eliminiere zunächst Zustand 4:
\\ \ \\
\begin{tikzpicture}[->, >=latex, node distance = 1cm, semithick]
\node[initial,state]	(START)						{Start};
\node[state]			(1) 	[right=2cm of START]{$1$};
\node[state]			(2) 	[right=2cm of 1]	{$2$};
\node[state]			(3) 	[below=2cm of 1]	{$3$};
\node[state,accepting]	(END) 	[right=2cm of 2]	{End};
s
\path 	(START) edge [above]					node {$\epsilon$} 	(1)
		(1) 	edge [above]					node {a} 			(2)
		(2)		edge [loop above, above]		node {b}			(2)
				edge [bend left=10, below right]node {c+ba}			(3)
				edge [above]					node {$\epsilon$}	(END)
		(3)		edge [bend left=10, above left]	node {a}		(2)
				edge [left]						node {b}			(1)
				edge [bend right =65, below]	node {$\epsilon$}	(END)
;
\end{tikzpicture}
\\ \ \\
%Schritt 3
Eliminiere Zustand 3:
\\ \ \\
\begin{tikzpicture}[->, >=latex, node distance = 1cm, semithick]
\node[initial,state]	(START)						{Start};
\node[state]			(1) 	[right=2cm of START]{$1$};
\node[state]			(2) 	[right=2cm of 1]	{$2$};
\node[state,accepting]	(END) 	[right=2cm of 2]	{End};
s
\path 	(START) edge [above]					node {$\epsilon$} 	(1)
		(1) 	edge [bend left=10, above]		node {a} 			(2)
		(2)		edge [loop above, above]		node {b+((c+ba)a)}	(2)
				edge [bend left=10, below ]		node {(c+ba)b}		(1)
				edge [above]					node {$\epsilon$+c+ba}	(END)		
;
\end{tikzpicture}
\\ \ \\
%Schritt 4
Eliminiere Zustand 2:
\\ \ \\
\begin{tikzpicture}[->, >=latex, node distance = 1cm, semithick]
\node[initial,state]	(START)						{Start};
\node[state]			(1) 	[right=2cm of START]{$1$};
\node[state,accepting]	(END) 	[right=2cm of 2]	{End};
s
\path 	(START) edge [above]					node {$\epsilon$} 	(1)
		(1) 	edge [above]					node {a(b+(c+ba)a)*($\epsilon$+c+ba)} 			(END)
				edge [loop below, below]		node {a(b+(c+ba)a)*(c+ba)b}				(1)		
;
\end{tikzpicture}
\\ \ \\
%Schritt 5
Eliminiere Zustand 1:
\\ \ \\
\begin{tikzpicture}[->, >=latex, node distance = 1cm, semithick]
\node[initial,state]	(START)						{Start};
\node[state,accepting]	(END) 	[right=3cm of 2]	{End};
s
\path 	(START) edge [above] node {(a(b+(c+ba)a)*(c+ba)b)*a(b+(c+ba)a)*($\epsilon$+c+ba)} 			(END)
;
\end{tikzpicture}
\\ \ \\
Somit erhalten wir also den regulären Ausdruck
\[\left(a\left(b+\left(c+ba\right)a\right)^{*}\left(c+ba\right)b\right)^{*}a\left(b+\left(c+ba\right)a\right)^{*}\left(\epsilon+c+ba\right)\]

%%%%%%%%%%%%%%%%%%%%%%%%%%%%%%%%%%%%%%%%%%%%%%%%%%%%%%%%%%%%%%%%%%%%%%%%%%%%%%%
%
%			Aufgabe H15
%
%%%%%%%%%%%%%%%%%%%%%%%%%%%%%%%%%%%%%%%%%%%%%%%%%%%%%%%%%%%%%%%%%%%%%%%%%%%%%%%
\paragraph{Aufgabe H15}
\begin{enumerate}[label=\alph*)]
\item Sei $L$ die Sprache mit Worten mit gleich vielen $a$'s und $b$'s.\\
\underline{Annahme: L regulär}\\
Sei $n\in\mathbb{N}$. Dann existieren $w\in L$ mit $\# a = \# b = n, \vert w \vert \geq 2n$.\\
Sei $w$ der Form $a^{n}b^{n}$.\\
Nach dem Pumping-Lemma kann man $w$ in $x,y,z$ zerlegen mit:
\begin{itemize}
\item $w=xyz$
\item $\vert xy \vert \leq n$
\item $\vert y \vert > 0$
\end{itemize}
Damit ist $y=a^{m}, 0<m\leq n$.\\
Nach dem Pumping-Lemma muss zudem gelten: $\forall i\in\mathbb{N}: xy^{i}z \in L$.\\
Betrachte $i=2$:
\begin{center}
$xy^2z=a^{n-m}a^{2m}b^{n}=a^{n+m}b^{n} \not\in L$, da $m>0 \Rightarrow n+m \neq n \Rightarrow \# a \neq \# b$.
\end{center}
Damit gilt das Pumping-Lemma nicht, weshalb $L$ nicht regulär ist.
\item $L=\{w\mid w\in \Sigma_{*}, \vert w \vert Fibonaccizahl\}$\\
\underline{Annahme: L regulär}\\
Sei $n \in\mathbb{N}$, betrachte $fib(n)$. Dann existieren $w\in\L$ mit $\vert w \vert \geq fib(n) + fib(n+1) = fib(n+2)$\\
Nach dem Pumping-Lemma kann man $w$ in $x,y,z$ zerlegen mit:
\begin{itemize}
\item $w=xyz$
\item $\vert xy \vert \leq n$
\item $\vert y \vert > 0$
\item $\forall i\in\mathbb{N}: xy^{i}z \in L$
\end{itemize}
$\Rightarrow 0<\vert y \vert \leq fib(n)$\\
Betrachte $i=0$:
\[fib(n+2)=fib(n+1)+fib(n)=\vert w \vert = \vert xyz \vert > \vert xy^0z \vert\] \[= \vert xyz \vert - \vert y \vert > fib(n+1) + fib(n) - fib(n) = fib(n+1)\]
$\Rightarrow fib(n+1) < \vert xy^0z \vert < fib(n+2)$\\$\Rightarrow \vert xy^0z \vert$ keine Fibonaccizahl\\$\Rightarrow xy^0z \not\in L$\\
Damit gilt das Pumping-Lemma nicht, weshalb $L$ nicht regulär ist.
\item $L=\{abca^nb^mcba\mid m\leq n\}$\\
\underline{Annahme: L regulär}\\
Sei $z\in\mathbb{N}$. Dann existieren Wörter $w\in L$ der Form $abca^nb^n$ mit $\vert w \vert \geq 2z+6, n \geq z$.
Nach dem Pumping-Lemma kann man $w$ in $x,y,z$ zerlegen mit:
\begin{itemize}
\item $w=xyz$
\item $\vert xy \vert \leq n$
\item $\vert y \vert > 0$
\item $\forall i\in\mathbb{N}: xy^{i}z \in L$
\end{itemize}
\begin{description}
\item[Fall 1:] $z=1$\par\nobreak
Damit muss $x=\epsilon, y=a$ gelten. Betrachte $i=0$. Dann ist $xy^iz=xy^0z=bca^nb^ncba\not\in L$.
\item[Fall 2:] $z=2$\par\nobreak
Analog zu [Fall 1]. $y$ ist entweder $a$, $ab$ oder $b$. Dann gilt auch für $i=0$ $xy^0z\not\in L$.
\item[Fall 3:] $z=3$\par\nobreak
Analog zu [Fall 2]. $y$ kann zusätzlich noch $abc$, $bc$ oder $c$ sein. Damit gilt auch hier für $i=0$ auch $xy^0z\not\in L$.
\item[Fall 4:] $z>3$\par\nobreak
\begin{description}
\item[Fall 4.1:] $0\leq \vert x \vert < 3$
Dann beinhaltet $y$ mindestens einer der ersten drei Buchstaben. Analog zu den Fällen zuvor gilt somit für $i=0$ auch $xy^0z\not\in L$.
\item[Fall 4.2:] $3\leq \vert x \vert < z$
Dann liegt $y$ in $a^n$, bzw. genauer in $a^{z-3}$. Sei $y$ also $a^j, 0<j\leq z-3$.\\
Betrachte auch hier $i=0$:
\[xy^iz=xy^0z=abca^{n-3-j}a^{0\cdot j}a^3b^ncba=abca^{n-j}b^ncba\]
Da $j>0 \Rightarrow \vert a^{n-j} \vert < \vert b^n \vert \Rightarrow xy^0z \not\in L$.
\end{description}
\end{description}
Damit ist gezeigt, dass das Pumping-Lemma nicht gilt. Daher ist $L$ keine reguläre Sprache.
\end{enumerate}


\end{document}