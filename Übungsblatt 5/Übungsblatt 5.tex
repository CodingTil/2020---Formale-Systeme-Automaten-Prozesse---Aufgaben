\documentclass[11pt]{article}

%Packages
\usepackage{amsfonts}	      %Mathematische Zeichen und Fonts
\usepackage{mathtools}        %Extra Mathematische Symbole
\usepackage{extarrows}	      %Extra Pfeile
\usepackage{listings}         %Codeansicht
\usepackage{scrlayer-scrpage} %Seitenkopf
\usepackage{tikz}             %tikz
\usepackage{enumitem}		  %Enumerate
\usepackage{listings}		  %Code snippets
\usepackage{amsmath}
\usepackage[normalem]{ulem}

\usetikzlibrary{arrows, automata, positioning}
\pagestyle{scrheadings}

\begin{document}

%%%%%%%%%% Header %%%%%%%%%%%%%%%%%%%%%%%%%%%%%%%%%%%%%%%%%%%%%%%%%%%%%%%%%%%%%
\ihead{\textbf{Formale Systeme, Automaten, Prozesse \\ Übungsblatt 5} \\Tutorium 11}
\ohead{Tim Luther, 410886 \\ Til Mohr, 405959\\ Simon Michau, 406133}

%%%%%%%%%% Seiteninhalt %%%%%%%%%%%%%%%%%%%%%%%%%%%%%%%%%%%%%%%%%%%%%%%%%%%%%%%

%%%%%%%%%%%%%%%%%%%%%%%%%%%%%%%%%%%%%%%%%%%%%%%%%%%%%%%%%%%%%%%%%%%%%%%%%%%%%%%
%
%			Aufgabe H13
%
%%%%%%%%%%%%%%%%%%%%%%%%%%%%%%%%%%%%%%%%%%%%%%%%%%%%%%%%%%%%%%%%%%%%%%%%%%%%%%%
\paragraph{Aufgabe H13}


%%%%%%%%%%%%%%%%%%%%%%%%%%%%%%%%%%%%%%%%%%%%%%%%%%%%%%%%%%%%%%%%%%%%%%%%%%%%%%%
%
%			Aufgabe H14
%
%%%%%%%%%%%%%%%%%%%%%%%%%%%%%%%%%%%%%%%%%%%%%%%%%%%%%%%%%%%%%%%%%%%%%%%%%%%%%%%
\paragraph{Aufgabe H14}
%Schritt 1
Wir schreiben die Endzustände als Zustände mit einer $\epsilon$-Transition zum Endzustand End, und den Startzustand als Zustand zu dem eine $\epsilon$-Transition vom Startzustand Start hinführt. 
\\ \ \\
\begin{tikzpicture}[->, >=latex, node distance = 1cm, semithick]
\node[initial,state]	(START)						{Start};
\node[state]			(1) 	[right=2cm of START]{$1$};
\node[state]			(2) 	[right=2cm of 1]	{$2$};
\node[state]			(3) 	[below=2cm of 1]	{$3$};
\node[state]			(4) 	[below=2cm of 2]	{$4$};
\node[state,accepting]	(END) 	[right=2cm of 2]	{End};
s
\path 	(START) edge [above]					node {$\epsilon$} 	(1)
		(1) 	edge [above]					node {a} 			(2)
		(2)		edge [right]					node {b}			(4)
				edge [loop above, above]		node {b}			(2)
				edge [bend left=10, below right]node {c}			(3)
				edge [above]					node {$\epsilon$}	(END)
		(3)		edge [bend left=10, above left]	node {a}		(2)
				edge [left]						node {b}			(1)
				edge [bend right =65, below]	node {$\epsilon$}	(END)
		(4)		edge [below]					node {a}			(3)
;
\end{tikzpicture}
\\ \ \\
%Schritt 2
Eliminiere zunächst Zustand 4:
\\ \ \\
\begin{tikzpicture}[->, >=latex, node distance = 1cm, semithick]
\node[initial,state]	(START)						{Start};
\node[state]			(1) 	[right=2cm of START]{$1$};
\node[state]			(2) 	[right=2cm of 1]	{$2$};
\node[state]			(3) 	[below=2cm of 1]	{$3$};
\node[state,accepting]	(END) 	[right=2cm of 2]	{End};
s
\path 	(START) edge [above]					node {$\epsilon$} 	(1)
		(1) 	edge [above]					node {a} 			(2)
		(2)		edge [loop above, above]		node {b}			(2)
				edge [bend left=10, below right]node {c+ba}			(3)
				edge [above]					node {$\epsilon$}	(END)
		(3)		edge [bend left=10, above left]	node {a}		(2)
				edge [left]						node {b}			(1)
				edge [bend right =65, below]	node {$\epsilon$}	(END)
;
\end{tikzpicture}
\\ \ \\
%Schritt 3
Eliminiere Zustand 3:
\\ \ \\
\begin{tikzpicture}[->, >=latex, node distance = 1cm, semithick]
\node[initial,state]	(START)						{Start};
\node[state]			(1) 	[right=2cm of START]{$1$};
\node[state]			(2) 	[right=2cm of 1]	{$2$};
\node[state,accepting]	(END) 	[right=2cm of 2]	{End};
s
\path 	(START) edge [above]					node {$\epsilon$} 	(1)
		(1) 	edge [bend left=10, above]		node {a} 			(2)
		(2)		edge [loop above, above]		node {b+((c+ba)a)}	(2)
				edge [bend left=10, below ]		node {(c+ba)b}		(1)
				edge [above]					node {$\epsilon$+c+ba}	(END)		
;
\end{tikzpicture}
\\ \ \\
%Schritt 4
Eliminiere Zustand 2:
\\ \ \\
\begin{tikzpicture}[->, >=latex, node distance = 1cm, semithick]
\node[initial,state]	(START)						{Start};
\node[state]			(1) 	[right=2cm of START]{$1$};
\node[state,accepting]	(END) 	[right=2cm of 2]	{End};
s
\path 	(START) edge [above]					node {$\epsilon$} 	(1)
		(1) 	edge [above]					node {a(b+(c+ba)a)*($\epsilon$+c+ba)} 			(END)
				edge [loop below, below]		node {a(b+(c+ba)a)*(c+ba)b}				(1)		
;
\end{tikzpicture}
\\ \ \\
%Schritt 5
Eliminiere Zustand 1:
\\ \ \\
\begin{tikzpicture}[->, >=latex, node distance = 1cm, semithick]
\node[initial,state]	(START)						{Start};
\node[state,accepting]	(END) 	[right=3cm of 2]	{End};
s
\path 	(START) edge [above] node {(a(b+(c+ba)a)*(c+ba)b)*a(b+(c+ba)a)*($\epsilon$+c+ba)} 			(END)
;
\end{tikzpicture}
\\ \ \\
Somit erhalten wir also den regulären Ausdruck
\[\left(a\left(b+\left(c+ba\right)a\right)^{*}\left(c+ba\right)b\right)^{*}a\left(b+\left(c+ba\right)a\right)^{*}\left(\epsilon+c+ba\right)\]

%%%%%%%%%%%%%%%%%%%%%%%%%%%%%%%%%%%%%%%%%%%%%%%%%%%%%%%%%%%%%%%%%%%%%%%%%%%%%%%
%
%			Aufgabe H15
%
%%%%%%%%%%%%%%%%%%%%%%%%%%%%%%%%%%%%%%%%%%%%%%%%%%%%%%%%%%%%%%%%%%%%%%%%%%%%%%%
\paragraph{Aufgabe H15}


\end{document}