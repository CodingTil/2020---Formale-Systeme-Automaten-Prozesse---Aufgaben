\documentclass[11pt]{article}

%Packages
\usepackage{amsfonts}	      %Mathematische Zeichen und Fonts
\usepackage{mathtools}        %Extra Mathematische Symbole
\usepackage{extarrows}	      %Extra Pfeile
\usepackage{listings}         %Codeansicht
\usepackage{scrlayer-scrpage} %Seitenkopf
\usepackage{tikz}             %tikz
\usepackage{enumitem}		  %Enumerate
\usepackage{listings}		  %Code snippets
\usepackage{amsmath}

\usetikzlibrary{arrows, automata, positioning}
\pagestyle{scrheadings}

\begin{document}

%Header
\ihead{\textbf{Formale Systeme, Automaten, Prozesse \\ Übungsblatt 2} \\Tutorium 11}
\ohead{Tim Luther, 410886 \\ Til Mohr, 405959\\ Simon Michau, 406133}

%Seiteninhalt
\paragraph{Aufgabe H7}
Der Aufruf der Methode \verb|check| gibt für einen regulären Ausdruck (\verb|RA|) $R$ zurück, ob dieser nur eine endliche Anzahl an Wörtern beschreibt. Folgender Algorithmus kann dies mit einer Worst-Case Laufzeit von $O(n)$ entscheiden, wobei $n$ die Anzahl der Teilausdrücke, die R aufbauen, ist:
\begin{lstlisting}[escapeinside=`']
bool check(RA R) {
	if(R `$\in \Sigma$') return true;
	if(R is A + B || R is AB) return check(A) && check(B);
	if(R is `$A^{*}$' || R is `$A^{+}$') return isEmpty(A);
}

bool isEmpty(RA R) {
	if(R is `$\epsilon$') return true;
	if(R `$\in \Sigma \setminus \{\epsilon\}$') return false;
	if(R is A + B `$\mid\mid$' R is AB) return isEmpty(A) && isEmpty(B);
}
\end{lstlisting}
$A$ und $B$ sind hier reguläre Ausdrücke, die hier $R$ darstellen können.

\paragraph{Aufgabe H8}
\begin{enumerate}[label=\alph*)]
\item 	Die Äquivalenzklassen sind $[a]_{\equiv _{L}} = \{a,b\}^{*}$ und $[c]_{\equiv _{L}} = \{a,b\}^{*}c\{a,b,c\}^{*}$. Erstere beschreibt alle Worte (keines aus L), bei denen man nur genau ein $c$ hinten anhängen kann, sodass sie immer noch in der Sprache $L$ sind. Die zweite Äquivalenzklasse beschreibt alle Worte (auch alle aus L), die mindestens ein $c$ haben, wodurch man nichts an das Wort anhängen kann, sodass es in der Sprache ist. Beide Äquivalenzklassen vereinigt bilden $\Sigma^{*}$:
\[[a]_{\equiv _{L}} \cup [c]_{\equiv _{L}} = \{a,b\}^{*} \cup \{a,b\}^{*}c\{a,b,c\}^{*} = \{a,b,c\}^{*} = \Sigma^{*}\]
Das leere Wort ist in $[a]_{\equiv _{L}}$ beinhaltet sowie alle Wörter, die kein $c$ enthalten. $[c]_{\equiv _{L}}$ beinhaltet alle Wörter, die mind. ein $c$ enthalten, an egal welcher Stelle. Die Vereinigung beider stellt also $\Sigma^{*}$ dar. 
\\Damit sind diese beiden Äquivalenzklassen die einzigen bzgl. der Myhill-Nerode Relation.

\item	
\end{enumerate}

\paragraph{Aufgabe H9}
Der Code ist beigefügt. Für die Aufgabe $\hat{\delta}(7,\verb|abababbaa|)$ fand das Programm nach ca. 289200ns das Ergebnis
\begin{center}
[32, 1, 2, 34, 3, 5, 6, 8, 9, 10, 11, 17, 22, 23, 25, 27, 28, 29, 31].
\end{center}

Für die 4 langen Wörter kam das Programm zu den Endzuständen:
\begin{center}
[32, 1, 34, 2, 3, 5, 6, 8, 9, 10, 11, 17, 20, 21, 25, 27, 28, 29, 31] (6336664800ns)

[32, 1, 34, 3, 4, 5, 6, 8, 9, 12, 13, 17, 18, 19, 25, 27, 29, 30, 31] (408921700ns)

[32, 1, 34, 3, 4, 5, 6, 8, 9, 17, 18, 19, 25, 27, 29, 30, 31] (2034853500ns)

[32, 1, 34, 3, 4, 5, 6, 8, 9, 17, 18, 19, 25, 27, 29, 30, 31] (21452262300ns)
\end{center}
\end{document}