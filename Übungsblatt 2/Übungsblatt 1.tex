\documentclass[11pt]{article}

%Packages
\usepackage{amsfonts}	      %Mathematische Zeichen und Fonts
\usepackage{mathtools}        %Extra Mathematische Symbole
\usepackage{extarrows}	      %Extra Pfeile
\usepackage{listings}         %Codeansicht
\usepackage{scrlayer-scrpage} %Seitenkopf
\usepackage{tikz}             %tikz

\usetikzlibrary{arrows, automata, positioning}
\pagestyle{scrheadings}

\begin{document}

%Header
\ihead{\textbf{Formale Systeme, Automaten, Prozesse \\ Übungsblatt 2} \\Tutorium 11}
\ohead{Tim Luther, 410886 \\ Til Mohr, 405959\\ Simon Michau, 406133}

%Seiteninhalt
\paragraph{Aufgabe H4}
%Produktautomat
\begin{center}
\begin{tikzpicture}[->, >=latex, node distance = 2cm, semithick]
\node[state]			(02)									{02};
\node[state]			(12) [right=2cm of 02]					{12};
\node[state,accepting]	(22) [right=2cm of 12]					{22};
\node[state]			(01) [below=2cm of 02]					{01};
\node[state]			(11) [right=2cm of 01]					{11};
\node[state]			(21) [right=2cm of 11]					{21};
\node[initial,state]	(00) [below=2cm of 01]					{00};
\node[state]			(10) [right=2cm of 00]					{10};
\node[state]			(20) [right=2cm of 10]					{20};

\path 	(02) 	edge [above] 				node {b} (12)%02
				edge [bend left=20, right] 	node {a} (01)
		(12) 	edge [loop above=20, above] node {b} (12)%12
				edge [bend left=20, right] 	node {a} (21)
		(22) 	edge [loop above=20, above] node {b} (22)%22
				edge [bend left=20, right] 	node {a} (11)
		(01) 	edge [above right] 			node {b} (10)%01
				edge [bend left=20, left] 	node {a} (02)
		(11) 	edge [right]			 	node {b} (10)%11
				edge [bend left=20, right] 	node {a} (22)
		(21) 	edge [right]			 	node {b} (20)%21
				edge [bend left=20, right] 	node {a} (12)
		(00) 	edge [above]			 	node {b} (10)%00
				edge [left] 				node {a} (01)
		(10) 	edge [loop below=20, below] node {b} (10)%10
				edge [right] 				node {a} (21)
		(20) 	edge [loop below=20, below]	node {b} (20)%20
				edge [left] 				node {a} (11)
;
\end{tikzpicture}
\end{center}

\paragraph{Aufgabe H5}
Seien $\delta_{A}$ und $\hat{\delta}_{A}$ die Übergangsfunktionen von A. $\hat{\delta}_{A}$ erkennt also Wörter w mit $w\in aL$ bzw. Wörter w mit $w=av, v\in L$. Sei $q_{0A}$ der Startzustand aus A. So kann man einen Automaten B mit Startzustand $q_{0B}$ bilden:
\[\hat{\delta}_{B}(q_{0B}, w) = \hat{\delta}_{A}(q_{0A}, aw) \forall w\in L\]
Da L eine reguläre Sprache ist und insbesondere A ein DFA ist, ist auch B ein DFA.

\newpage
\paragraph{Aufgabe H6}
%Potenzmengen DFA
\begin{center}
\begin{tikzpicture}[->, >=latex, node distance = 2cm, semithick]
\node[initial,state]	(1) 					{\{1\}};
\node[state,accepting]	(3)	[right=2cm of 1] 	{\{3\}};
\node[state,accepting] 	(13)[right=2cm of 3] {\{1,3\}};
\node[state,accepting]	(2) [above=2cm of 13] {\{2\}};
\node[state,accepting]	(23)[right=2cm of 13] {\{2,3\}};

\path	(1) edge [above]				node {a,b} 	(3)
		(3) edge [below]				node {b} 	(13)
			edge [above]				node {a} 	(2)
		(2) edge [loop right, right]	node {a,b}	(2)
		(13)edge [bend left=20, above] 	node {a} 	(23)
			edge [loop below, below] 	node {b} 	(13)
		(23)edge [loop right, right] 	node {a} 	(23)
			edge [bend left=20, below] 	node {b} 	(13)
;
\end{tikzpicture}
\end{center}

\end{document}